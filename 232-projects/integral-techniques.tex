\documentclass[12pt]{article}
\usepackage[fleqn]{amsmath}     %puts eqns to left, not centered
\usepackage{graphicx}
\usepackage{hyperref}
\begin{html}
<style>
pre {font-size: 1.2em; background-color: #EEF0F5;}
ul li {list-style-image: url(http://www.math.csi.cuny.edu/static/images/julia.png);}  
</style>
\end{html}
\begin{document}
\section{Questions to be handed in for project Techniques of Integration}\newline
Begin by loading our package for plotting our package that allows symbolic math:\begin{verbatim}
using Plots; gadfly()
using SymPy
\end{verbatim}
\newline
This project covers some of the techniques of integration utilizing the symbolic algebra capabilities provide by the \texttt{SymPy} package. \newline
The basic techniques of integration involve \begin{itemize}\item substitution (reverse chain rule)\item integration by parts (reversed product rule)\item partial fractions (for rational functions)\end{itemize}\newline
The \texttt{SymPy} function \texttt{integrate} is a implementation of the \href{http://en.wikipedia.org/wiki/Risch_algorithm}{Risch Algorithm}.  This algorithm can decide if a function comprised of elementary functions has an antiderivative expressed in elementary function.  For example, neither $\sin(x^2)$ or $e^{x^2}$ has such a "nice" antiderivative, whereas $\sin(x)^2$ and $xe^{-x^2}$ do. Risch's algorithm exploits \href{http://asmeurersympy.wordpress.com/2010/08/}{Liouville's Theorem} which gives a representation for any potential antiderivative of $f(x)$ using \textit{elementary functions} such as polynomials, logarithms, exponentials and trigonometric functions.\subsubsection{Questions}\subsubsection{When Base.Markdown.Code("","integrate") works}\begin{itemize}\item use \texttt{integrate} to find an antiderivative for \end{itemize}
$$
\int \cos(x) \log(\sin(x)) dx.
$$
\newline
Judging from the answer does it appear to use parts, substitution, both, or something else.\begin{itemize}\item use \texttt{integrate} to find an antiderivative for \end{itemize}
$$
\int (\log(x))^2 dx.
$$
\newline
Judging from the answer does it appear to use parts, substitution, both, or something else.\begin{itemize}\item use \texttt{integrate} to find an antiderivative for \end{itemize}
$$
\int \sin(x) \cdot e^x dx.
$$
\newline
Judging from the answer does it appear to use parts, substitution, both, or something else.\begin{itemize}\item The \texttt{integrate} function might fail on\end{itemize}
$$
\int x \sec^2(x) dx.
$$
\newline
But, replacing $\sec(x)$ with $1/\cos(x)$ may work. Does it?\subsubsection{It doesn't always work ...}\newline
The \texttt{integrate} function works for many functions, but for some one can help by performing substitutions by hand first.\newline
Here is an example from \href{http://faculty.uml.edu/jpropp/142/Integration.pdf}{here} where performing the substitution by hand is needed to give a boost. Find
$$
\int (1 + \log(x)) \cdot \sqrt{1 + (x\cdot\log(x))^2} dx.
$$
\begin{verbatim}
x, dx = symbols("x, dx")
f(x) = sqrt(x^2 + 1)
u = x*log(x)
ex = f(u) * diff(u,x)
\end{verbatim}
$$\sqrt{x^{2} \log^{2}{\left (x \right )} + 1} \left(\log{\left (x \right )} + 1\right)$$\newline
This is written to look like a substitution problem. Will \texttt{SymPy} be able to see that?\begin{verbatim}
integrate(ex, x)
\end{verbatim}
$$\int \sqrt{x^{2} \log^{2}{\left (x \right )} + 1} \left(\log{\left (x \right )} + 1\right)\, dx$$\newline
Sympy can't do this integral, and so returns the expression. We can help.\newline
Let's introduce a variable \texttt{dx} into our expression, and then we have:\begin{verbatim}
ex =  f(u) * diff(u,x) * dx
\end{verbatim}
$$dx \sqrt{x^{2} \log^{2}{\left (x \right )} + 1} \left(\log{\left (x \right )} + 1\right)$$\newline
Then we can have this substitution $u = x\log(x)$  Then $du = (x \cdot \log(x))' dx$.\newline
We replace each $x \log(x)$ with a \texttt{u}, and replace $dx$ with $dy/(x \cdot \log(x))'$. The \texttt{subs} function can do so:\begin{verbatim}
u, du = symbols("u, du")
## replace x*log(x) with u and diff(x*log(x),x) * dx with du
ex1 = subs(ex, (x*log(x), u), (dx, du/diff(x*log(x),x)))
\end{verbatim}
$$du \sqrt{u^{2} + 1}$$\newline
This latter function can be integrated (after stripping off the differential we added for familiarity):\begin{verbatim}
ex = integrate(ex1 / du)
\end{verbatim}
$$\frac{u}{2} \sqrt{u^{2} + 1} + \frac{1}{2} \operatorname{asinh}{\left (u \right )}$$\newline
If needed we can resubstitute in $x \log(x)$ for $u$ to get an answer.\begin{verbatim}
subs(ex, (u, x*log(x)))
\end{verbatim}
$$\frac{x}{2} \sqrt{x^{2} \log^{2}{\left (x \right )} + 1} \log{\left (x \right )} + \frac{1}{2} \operatorname{asinh}{\left (x \log{\left (x \right )} \right )}$$\newline
For future use, we codify the above steps in a function:\begin{verbatim}
function usub(ex, let_u_equal)
    u, du, dx = symbols("u, du, dx")
    ex1 = ex * dx
    ex2 = subs(ex1, (dx, du/diff(let_u_equal, x)))
    ex3 = subs(ex2, (let_u_equal, u))
    ex3 / du
end
\end{verbatim}
\begin{verbatim}
usub (generic function with 1 method)\end{verbatim}
\newline
This can be used as follow:\begin{verbatim}
ex = log(x)/x
usub(ex, log(x))
\end{verbatim}
$$u$$\newline
While we are here, we also give this quick function for integration by parts. Just pick "u" and let \texttt{SymPy} do the ret:\begin{verbatim}
function udv_parts(ex, let_u_equal)
    u, dv = let_u_equal, ex/let_u_equal
    du = diff(u, x)
    v = integrate(dv)
    [u*v, v*du]			# return two pieces. One for FTC, one to integrate
end

ex = x*sin(x)
uv, vdu = udv_parts(ex, sin(x))
\end{verbatim}
\begin{bmatrix}\frac{x^{2}}{2} \sin{\left (x \right )}\\\frac{x^{2}}{2} \cos{\left (x \right )}\end{bmatrix}\subsubsection{Questions}\subsubsection{The absolute value}\begin{itemize}\item Does \texttt{integrate} know how to integrate $|x|$?\end{itemize}\begin{itemize}\item Assert to \texttt{SymPy} that $x > 0$ by defining it via:\end{itemize}\begin{verbatim}
x = symbols("x", positive=true)
\end{verbatim}
$$x$$\newline
Does \texttt{integrate} now know how to integrate $|x|$?\begin{itemize}\item What is a simple antiderivative for $\int |x| dx$?\end{itemize}\subsubsection{Helping Base.Markdown.Code("","integrate") out}\begin{itemize}\item Does \texttt{integrate}  find an antiderivative for \end{itemize}
$$
\int \log(\log(x)) / x dx?
$$
\newline
If not, help it out.\begin{itemize}\item The following integral exists, but is not found by \texttt{integrate}:\end{itemize}
$$
\int \frac{(-\log(x)^2 + 1)^{1/2}}{x} dx
$$
\newline
Make a $u$-substitution to help \texttt{integrate} out with finding an answer:\begin{itemize}\item The following integral exists, but is not found by \texttt{integrate}:\end{itemize}
$$
\int \frac{1}{x \sqrt{1 - \log(x)^2}}dx.
$$
\newline
Make a $u$-substitution to help \texttt{integrate} out with finding an answer:\begin{itemize}\item What do you get if you try a "u"-"dv" integration by parts for $\int   e^x \sin(x) dx$?  Does it seem to help find the integral?\end{itemize}\begin{itemize}\item The \href{http://en.wikipedia.org/wiki/Integration_by_parts#LIATE_rule}{LIATE rule} is a rule of thumb for identifying what $u$ should be in integrating by parts. Apply it to \end{itemize}
$$
\int x^2 e^x dx.
$$
\newline
What is $u$? What is the "$v \cdot du$" term? Does it seem "easier" than what you started with? Why?\subsubsection{Integration of rational functions:}\newline
Some background reading on the implementation is \href{http://asmeurersympy.wordpress.com/2010/06/11/integration-of-rational-functions/}{here}\newline
A \textit{rational function} is a ratio of polynomial functions. Using polynomial long division and dividing out common factors, up to removable singularities, one can uniquely write a rational function in terms of three other polynomials:
$$
a(x)/b(x) =  s(x) + r(x)/q(x)
$$
\newline
where the degree of $r(x)$ is less than the degree of $q(x)$ and there are no common roots to $r(x)$ and $q(x)$. It is easy to integrate $s(x)$. What about the $r(x)/q(x)$?\newline
We know from the \textit{fundamental theorem of algebra} that we can factor $q(x) = p_1(x)^n_1 \cdots p_k(x)^{n_k}$, where $p_i$ is a linear or quadratic factor. Further, the \href{http://en.wikipedia.org/wiki/Partial_fraction_decomposition}{partial fraction decomposition} ensures then that the ratio can be written as:
$$
\frac{r(x)}{q(x)} = \sum_i^k \frac{a_{i1}(x)}{p_i(x)} + \frac{a_{i2}(x)}{p_i(x)^2} + ... + \frac{a_{in_i}(x)}{p_i(x)^{n_i}}
$$
\newline
with the $a_{ij}(x)$ being polynomials of degree less than or equal the $p_i(x)$, so in this case either a constant or linear polynomial.\newline
So, if the polynomials of the form $(ax+b)/(cx^2 + dx + e)^j$ can be integrated, by the linearity of integration all rational functions can be integrated. Let's investigate this question.\subsubsection{Questions}\begin{itemize}\item The \texttt{apart} function separates rational functions into pieces. Use   \texttt{apart} to find the partial fraction decomposition of\end{itemize}
$$
f(x) = \frac{1}{x(x+1)}
$$
\begin{itemize}\item  Use \texttt{apart} to find the partial fraction decomposition of \end{itemize}
$$
f(x) = \frac{1}{x(x+1)^5}
$$
\begin{itemize}\item  Use \texttt{apart} to find the partial fraction decomposition of \end{itemize}
$$
f(x) = \frac{1}{x(x^2+1)}
$$
\begin{itemize}\item  Use \texttt{apart} to find the partial fraction decomposition of \end{itemize}
$$
f(x) = \frac{1}{x(x^2+1)^3}
$$
\begin{itemize}\item  Use \texttt{apart} to find the partial fraction decomposition of \end{itemize}
$$
f(x) = \frac{1}{(x^2+2)^2 (x^2+3)^3}
$$
\begin{itemize}\item Let $m$ be a positive integer. For which values of $m$ will \end{itemize}
$$
\int \frac{1}{x^m} dx
$$
\newline
will have a rational function for an answer? What is the answer when it isn't a rational function?\begin{itemize}\item What is the integral of a term like:\end{itemize}
$$
\int \frac{1}{x^2 + 1} dx
$$
\begin{itemize}\item What is the integral of a term like\end{itemize}
$$
\int \frac{x}{x^2 + 1} dx
$$
\begin{itemize}\item For non-negative, integer values of $m$, do these integrals appear   to always be rational functions? (Just try some different values   starting with $m=1$.)\end{itemize}
$$
\int \frac{1}{(x^2+1)^m} dx
$$
\begin{itemize}\item For non-negative, integer values of $m$, do these integrals appear   to always be rational functions?\end{itemize}
$$
\int \frac{x}{(x^2+1)^m} dx
$$
\newline
Verify this is the case for any $m$ by integrating the above symbolically. (You might also try to do the substitution manually.)
\end{document}
