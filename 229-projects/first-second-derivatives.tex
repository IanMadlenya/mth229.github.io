\documentclass[12pt]{article}
\usepackage[fleqn]{amsmath}     %puts eqns to left, not centered
\usepackage{graphicx}
\usepackage{hyperref}
\begin{html}
<style>
pre {font-size: 1.2em; background-color: #EEF0F5;}
ul li {list-style-image: url(http://www.math.csi.cuny.edu/static/images/julia.png);}  
</style>
\end{html}
\begin{document}

\section{Questions to be handed in for project 7:}

To get started, we load \texttt{Gadfly} so that we can make plots, and
load the \texttt{Roots} package for \texttt{D} and \texttt{fzero}:



\begin{verbatim}
using Gadfly            
using Roots         
\end{verbatim}
\subsubsection{Quick background}

Read about this material here:
\href{http://mth229.github.io/first-second-derivatives.html}{Exploring
first and second derivatives with Julia}.

For the impatient, this assignment looks at the relationship between a
function, $f(x)$, and its first and second derivatives, $f'(x)$ and
$f''(x)$. The basic relationship can be summarized (though the devil is
in the details) by:

\begin{enumerate}
\def\labelenumi{\arabic{enumi})}
\item
  if the first derivative is \emph{positive} on $(a,b)$ then the
  function is \emph{increasing} on $(a,b)$.
\item
  If the second derivative is \emph{positive} on $(a,b)$ then the
  function is \emph{concave up} on $(a,b)$.
\end{enumerate}

(The devil here is that the converse statements are not quite always
true.)

As well, use the definition of a \emph{critical point} being a value in
the domain of $f(x)$ for which the derivative is $0$ or undefined.

We can use the \texttt{D} operator from the \texttt{Roots} package to
find the first and second derivatives of \texttt{f}. The usage follows
this pattern:



\begin{verbatim}
f(x) = sin(x)           # some function
fp(x) = D(f)(x)         # makes fp the first derivative, though using D(f) is maybe better.
fpp(x) = D(f,2)(x)      # makes fpp the second derivative, though using D(f,2) is maybe better.
\end{verbatim}
In the notes, the following function is used to plot a function
\texttt{f} two ways: once as usual, the second time showing the function
\texttt{f} \emph{only if} the function \texttt{g} is positive.



\begin{verbatim}
function plotif(f, g, a, b)
  plot([f, x -> g(x) > 0.0 ? f(x) : NaN], a, b)
end
\end{verbatim}
This allows a graphical exploration of the above facts:



\begin{verbatim}
plotif(f, D(F), 0, 2pi)     # shows sin(x) and when its derivative is 0
\end{verbatim}
Clicking on the \texttt{f1} in the legend on the right will hide the
entire graph of \texttt{f} and leave only the graph of \texttt{f} where
\texttt{fp(x) \textgreater{} 0}. That is, where $f$ is known to be
increasing by the basic fact above.



Many questions resolve to finding zeros of some function. The
\texttt{fzeros} function can be useful for this -{}- but it only finds
simple roots. A repeated use of \texttt{fzero} with good initial guesses
is more reliable. For example, from the graph of
$f(x) = 1 + \cos(x) + \cos(2x)$, we can see in $[0,2\pi]$ that there are
zeros \emph{near} 1.5, 2, 4 and 5. These can be improved with:



\begin{verbatim}
f(x) = 1 + cos(x) + cos(2x)
guesses = [1.5, 2, 4, 5]
[fzero(f, x) for x in guesses] # do 4 things at once
\end{verbatim}
In this case, you know that the zeros are all \emph{simple zeros} so
this could have used: \texttt{fzeros(f, 0, 6)}.

\subsubsection{Questions, Graphical approach}

\begin{itemize}
\itemsep1pt\parskip0pt\parsep0pt
\item
  Make a graph using \texttt{plotif} to investigate when the
  \texttt{airy} function is positive on the interval $(-5,5)$. Your
  answer should use interval notation.
\end{itemize}

\begin{answer}
type: longtext
reminder: Commands to plot airy to see when it is positive
answer_text: (-5,-4.08) and (-2.33, 5) 
rows: 3
cols: 60
\end{answer}

\begin{itemize}
\itemsep1pt\parskip0pt\parsep0pt
\item
  Make a graph using \texttt{plotif} to investigate when the function
  $f(x) =   x^x$ is \emph{increasing} on the interval $(0,2)$. Your
  answer should use interval notation.
\end{itemize}

\begin{answer}
type: longtext
reminder: Interval(s) in (0,2) where \( x^x \) is increasing.
answer_text: Roughly (0.3678, 2) 
rows: 3
cols: 60
\end{answer}

\begin{itemize}
\itemsep1pt\parskip0pt\parsep0pt
\item
  Make a graph using \texttt{plotif} to investigate when the function
\end{itemize}

\[
f(x) = \frac{x}{x^2+9}
\]

is \emph{concave up} on the interval $(-10,10)$. Your answer should use
interval notation.

\begin{answer}
type: longtext
reminder: When on (-10, 10) is \( x/(x^2 + 9) \) concave up?
answer_text: (-5.196, 0), (5.196, oo) 
rows: 3
cols: 60
\end{answer}

\begin{itemize}
\itemsep1pt\parskip0pt\parsep0pt
\item
  Make a graph using \texttt{plotif} to identify any \emph{critical
  points} of $f(x) = x \ln(x)$ on the interval $(0,4)$.
\end{itemize}

\begin{answer}
type: longtext
reminder: Critical point of \( x \log(x) \) on (0,4)
answer_text: 0.367879 is only one 
rows: 3
cols: 60
\end{answer}

\begin{itemize}
\itemsep1pt\parskip0pt\parsep0pt
\item
  Make a graph using \texttt{plotif} to identify any \emph{inflection
  points} of $f(x) = \sin(x) - x$ over the interval $(-5,5)$.
\end{itemize}

\begin{answer}
type: longtext
reminder: inflection points of sin(x) - x in (-5,5)
answer_text: -pi, 0, pi 
rows: 3
cols: 60
\end{answer}

\begin{itemize}
\itemsep1pt\parskip0pt\parsep0pt
\item
  For a polynomial $p(x)$ between any two zeros there must be a critical
  point, and perhaps more than one. Verify graphically this is the case
  for $p(x) =x^4 + x^3 - 7x^2 - x + 6$. What commands do you use? How do
  these commands verify this case?
\end{itemize}

\begin{answer}
type: longtext
reminder: Verify that between each zero of \( p(x) \) is at least one zero of \( p'(x) \)
answer_text: Compare, say, \verb+fzeros(p)+ with \verb+fzeros(D(p), -4, 3)+ 
rows: 3
cols: 60
\end{answer}

\subsubsection{Finding numeric values}

As a reminder

\begin{itemize}
\item
  a \emph{critical} point of $f$ is a value in the domain of $f(x)$ for
  which the derivative is $0$ or undefined. These are often but
  \textbf{not always} where $f(x)$ has a local maximum or minimum.
\item
  An \emph{inflection point} of $f$ is a value in the domain of $f(x)$
  where the concavity of $f$ changes. (These are \emph{often} but
  \textbf{not always} where $f''(x)=0$.)
\end{itemize}

We can graphically identify \emph{critical points} of $f(x)$ by graphing
the function's derivative and looking for when the derivative is 0 or
undefined. Numerically, we can locate values where the derivative
crosses $0$ using the \texttt{fzero} function from the \texttt{Roots}
package.

\begin{itemize}
\itemsep1pt\parskip0pt\parsep0pt
\item
  Use \texttt{fzero} to numerically identify all \emph{critical points}
  to the function $f(x) = 2x^3 - 6x^2 - 2x + 4$. (There are no more than
  $2$.)
\end{itemize}

\begin{answer}
type: longtext
reminder: critical points of \( f(x) = 2x^3 - 6x^2 - 2x + 4 \)
answer_text: \verb+fzeros(D(f), -10,10)+ gives -0.15455, 2.1547 
rows: 3
cols: 60
\end{answer}

\begin{itemize}
\itemsep1pt\parskip0pt\parsep0pt
\item
  Use \texttt{fzero} to numerically identify all \emph{inflection
  points} for the function $f(x) = \ln(x^2 + 2x + 5)$.
\end{itemize}

\begin{answer}
type: radio
reminder: all inflection points
values: 1 | 2 | 3 | 4
labels: There are none | There is one at x=-1.0 | There is one at x=1.0 and one at x=-3.0 | There is one at each of x=-4.4641, -1.0, and 2.4641
answer: 3
\end{answer}

\begin{itemize}
\itemsep1pt\parskip0pt\parsep0pt
\item
  Suppose $f'(x) = x^3 - 6x^2 + 11x - 6$. Where is $f(x)$ increasing?
  Use interval notation in your answer.
\end{itemize}

\begin{answer}
type: radio
reminder: When is f(x) increasing?
values: 1 | 2 | 3 | 4
labels: It is always increasing | (-oo, 1.42265) and (2.57735, oo) | (1.0, 2.0) and (3.0, oo) | (2.0, oo)
answer: 3
\end{answer}

\begin{itemize}
\itemsep1pt\parskip0pt\parsep0pt
\item
  Suppose $f''(x) = x^2 - 3x + 2$. Where is $f(x)$ concave up? Use
  interval notation in your answer.
\end{itemize}

\begin{answer}
type: radio
reminder: When is f(x) concave up?
values: 1 | 2 | 3 | 4
labels: (-oo, oo) -- it is always concave up | (1.5, oo) | (1.0, 2.0) | (-oo, 1.0) and (2.0, oo)
answer: 4
\end{answer}

\subsubsection{The derivative tests}

\paragraph{The first derivative test}

This states that for a differentiable function $f(x)$ with a critical
point at $c$ then if $f'(x)$ changes sign from $+$ to $-$ at $c$ then
$f(c)$ is a local maximum and if it changes sign from $-$ to $+$ then
$f(c)$ is a local minimum.

\begin{itemize}
\itemsep1pt\parskip0pt\parsep0pt
\item
  For the function $f(x)$ suppose you know $f'(x)=x^3 - 5x^2 + 8x - 4$.
  Find \emph{all} the critical points. Use the first derivative test to
  classify them as local extrema \emph{if} you can. If you can't say
  why.
\end{itemize}

\begin{answer}
type: longtext
reminder: Find critical points of f(x), when \( f'(x)=x^3 - 5x^2 + 8x -4 \). Classify
answer_text: 1,2 are cps, changes sign - to + at 1, so a min; does not change sign at 2, not an extrema 
rows: 3
cols: 60
\end{answer}

\paragraph{The second derivative test}

This states that if $c$ is a critical point of $f(x)$ and $f''(c) > 0$
then $f(c)$ is a local minimum and if $f''(c) < 0$ then $f(c)$ is a
local maximum.

\begin{itemize}
\itemsep1pt\parskip0pt\parsep0pt
\item
  Suppose $f'(x) = (x^2 - 2) \cdot e^{-x}$. First find the critical
  points of $f(x)$, then use the second derivative test to classify
  them.
\end{itemize}

The critical points are:

\begin{answer}
type: radio
reminder: critical points
values: 1 | 2 | 3 | 4
labels: -1.41421, 1.41421 | 0.0 | -0.732051 | -0.732051, 2.73205
answer: 1
\end{answer}

Classify your critical points using the second derivative test

\begin{answer}
type: longtext
reminder: Classify critical points with second derivative test
answer_text: \verb+[D(fp)(x) for x in sort(fzeros(fp, -5,5))]+ indicates max at -sqrt(2) and min at sqrt(2) 
rows: 3
cols: 60
\end{answer}

\begin{itemize}
\itemsep1pt\parskip0pt\parsep0pt
\item
  Suppose $f'(x) = x^3 - 7x^2 + 14$. Based on the plot over the interval
  $[-4, 8]$. On what subintervals is $f(x)$ increasing?
\end{itemize}

\begin{answer}
type: radio
reminder: when is f increasing
values: 1 | 2 | 3 | 4
labels: (-1.29, 1.61) and (6.69, oo) | (-oo, 0) and (4.67, oo) | (-oo, 0) | (-oo, 00) and (6.69, oo)
answer: 1
\end{answer}

What did you use to find your last answer?

\begin{answer}
type: radio
reminder: how?
values: 1 | 2 | 3 | 4
labels: f'(x) > 0 on these subintervals | f''(x) > 0 on these subintervals | f'(x) < 0 on these subintervals | f''(x) < 0 on these subintervals
answer: 1
\end{answer}

What are the $x$-coordinates of the relative minima of $f(x)$?

\begin{answer}
type: radio
reminder: where are minima?
values: 1 | 2 | 3 | 4
labels: 4.56 | -1.29 and 6.69 | 4.56 and 0 | -1.29 and 1.61
answer: 2
\end{answer}

On what subintervals is $f(x)$ concave up?

\begin{answer}
type: radio
reminder: Where is f(x) concave up?
values: 1 | 2 | 3 | 4
labels: (-oo, 0) and (4.67, oo) | (1.167, oo) | (-oo, 1.167) | It is always concave down
answer: 1
\end{answer}

What did you use to decide?

\begin{answer}
type: radio
reminder: Where is f(x) concave up?
values: 1 | 2 | 3 | 4
labels: f'(x) > 0 on these subintervals | f''(x) > 0 on these subintervals | f'(x) < 0 on these subintervals | f''(x) < 0 on these subintervals
answer: 2
\end{answer}

Find the $x$ coordinates of the inflection points of $f(x)$.

\begin{answer}
type: radio
reminder: inflection points
values: 1 | 2 | 3 | 4
labels: 4.56 | 4.67 | 0 and 4.67 | Not listed
answer: 3
\end{answer}

\begin{itemize}
\itemsep1pt\parskip0pt\parsep0pt
\item
  A simplified model for the concentration of a certain slow-reacting
  medicine in the bloodstream $t$ hours after injection into muscle
  tissue is given by
\end{itemize}

\[
f(t) =  3t^2 \cdot e^{-t/5}, quad t \geq 0.
\]

When will there be maximum concentration?

\begin{answer}
    type: numeric
    reminder: when is max. concentration?
    answer: [9.9999, 10.0001]

\end{answer}

In the units given, how much is the maximum concentration?

nothing
When will the concentration dip below a level of 1.0?

\begin{answer}
    type: numeric
    reminder: when will it dip below 1.0?
    answer: [43.12683391715227, 43.146833917152264]

\end{answer}

Estimate from a graph when the concentration function changes concavity:

\begin{answer}
type: longtext
reminder: When does concentration change concavity?
answer_text: At 2.92 and 17.07, e.g. fzeros(D(f,2), 0, 30) 
rows: 3
cols: 60
\end{answer}

\subsubsection{Concave up has alternate definitions}

The Rogawski book defines $f(x)$ to be concave up for differentiable
functions by:

\begin{quote}
$f(x)$ is concave up on $(a,b)$ if $f'(x)$ is increasing on $(a,b)$.
\end{quote}

More generally, one can define $f(x)$ as concave up on $(a,b)$ if for
any pair of points, $c$ and $d$ with $a < c < d < b$ one has the secant
line connecting $(c,f(c))$ and $(d,f(d))$ lies \emph{above} the graph of
$f(x)$ over $(c,d)$.

For the function $f(x) = x^2 - 2x$, graphically verify this is the case
for 3 pairs of points between $(-5,5)$. The following can be used to
create a function for a secant line between $c$ and $d$:



\begin{verbatim}
function secline(f, c, d)
  x0, y0, m = c, f(c), (f(c) - f(d)) / (c - d)
  x -> y0 + m * (x - x0)    # pt-slope form as function
end
\end{verbatim}
\begin{answer}
type: longtext
reminder: Commands to produce graph of \( x^2 - 2x \) and three secant lines

rows: 3
cols: 60
\end{answer}

\begin{itemize}
\itemsep1pt\parskip0pt\parsep0pt
\item
  For the function $f(x) = x^3 - 2x$ find a pair of points, $c$ and $d$,
  in $(-3,3)$ which illustrate that the function is not concave up.
\end{itemize}

\begin{answer}
type: longtext
reminder: Commands to produce graph of \( x^3 - 2x \) and a secant line that is not always above graph

rows: 3
cols: 60
\end{answer}

\end{document}

