\documentclass[12pt]{article}
\usepackage[fleqn]{amsmath}     %puts eqns to left, not centered
\usepackage{graphicx}
\usepackage{hyperref}
\begin{html}
<style>
pre {font-size: 1.2em; background-color: #EEF0F5;}
ul li {list-style-image: url(http://www.math.csi.cuny.edu/static/images/julia.png);}  
</style>
\end{html}
\begin{document}

\subsection{Extra credit problems on related rates}

Load some packages:



\begin{verbatim}
using Gadfly
using Roots
using SymPy
\end{verbatim}
Related rates problems are questions where two (or more) unknown
quantities are related through an equation, hence their rates - change
with respect to some variable which is often time -{}- are related.

Here is an example from
\href{http://oregonstate.edu/instruct/mth251/cq/Stage9/Practice/ratesProblems.html}{here}

\begin{quote}
A screen saver displays the outline of a 3 cm by 2 cm rectangle and then
expands the rectangle in such a way that the 2 cm side is expanding at
the rate of 4 cm/sec and the proportions of the rectangle never change.
How fast is the area of the rectangle increasing when its dimensions are
12 cm by 8 cm?
\end{quote}

We begin by identifying the area of the rectangle depends on the
dimensions, in this case through the formula



\begin{verbatim}
A(w, h) = w * h
\end{verbatim}
The size of the height as a function of $t$ is



\begin{verbatim}
w(t) = 2 + 4*t
\end{verbatim}
As the proportions never change, we have



\begin{verbatim}
h(t) = 3/2 * w(t)
\end{verbatim}
Now to answer the question, when the width is 8, we must have that $t$
is:



\begin{verbatim}
t = fzero(x -> w(x) - 8, [0, 4])        # or solve by hand
\end{verbatim}
Then the key here is the rate of change of area depends on the rate of
change of h and w. We can either use the chain rule to get this:

\[
dA/dt = (dw/dt) \cdot h + w \cdot (dh/dt)
\]

which can be solved, as we get $dw/dt = 4$ and
$dh/dt = (3/2)dw/dt = (3/2) \cdot 4 = 6$. All told then the answer is
$4\cdot 12 + 8 \cdot 6 = 96$

However, here we can express $A$ as a function of $t$ by composition,
then differentiate that:



\begin{verbatim}
D(t -> A(w(t), h(t))) (t)
\end{verbatim}
For this, you might have been able to see the answer graphically:



\begin{verbatim}
plot(A(w(x), h(x)), 0, 3)
\end{verbatim}
That example is on the "easier" side, as it requires no algebra to solve
for the answer, however it is on the "harder" side as there are two
functions (\texttt{w} and \texttt{h}) that one must differentiate with
respect to \texttt{t}. Here are some more.

\begin{center}\rule{3in}{0.4pt}\end{center}

*

\begin{quote}
An FBI agent with a powerful spyglass is located in a boat anchored 400
meters offshore. A gangster under surveillance is walking along the
shore. Assuming the shoreline is straight and that the gangster is
walking at the rate of 2 km/hr, how fast must the FBI agent rotate the
spyglass to track the gangster when the gangster is 1 km from the point
on the shore nearest to the boat. Convert your answer to degrees/minute.
\end{quote}

*

\begin{quote}
A flood lamp is installed on the ground 200 feet from a vertical wall. A
six foot tall man is walking towards the wall at the rate of 30 feet per
second. How fast is the tip of his shadow moving down the wall when he
is 50 feet from the wall?
\end{quote}

*

\begin{quote}
A receptacle is in the shape of an inverted square pyramid 10 inches in
height and with a 6 x 6 square base. The volume of such a pyramid is
given by
\end{quote}

\[
\frac{1}{3}(\text{area of base}) \cdot (\text{height})
\]

\begin{quote}
Suppose that the receptacle is being filled with water at the rate of .2
cubic inches per second. How fast is water rising when it is 2 inches
deep? (See the
\href{http://oregonstate.edu/instruct/mth251/cq/Stage9/Practice/ratesProblems.html}{figure})
\end{quote}

*

\begin{quote}
Consider the hyperbola y = 1/x and think of it as a slide. A particle
slides along the hyperbola so that its x-coordinate is increasing at a
rate of f(x) units/sec. If its y-coordinate is decreasing at a constant
rate of 1 unit/sec, what is f(x)? (cf. figure)
\end{quote}

*

\begin{quote}
Two runners are running on circular tracks each of which has a
circumference of 1320 feet. The tracks are 100 feet apart and the
runners start opposite each other and move at the same constant rate of
880 ft/min. How fast are the runners separating when each has run 165
feet?
\end{quote}

\end{document}

