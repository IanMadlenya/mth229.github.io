\documentclass[12pt]{article}
\usepackage[fleqn]{amsmath}     %puts eqns to left, not centered
\usepackage{graphicx}
\usepackage{hyperref}
\begin{html}
<style>
pre {font-size: 1.2em; background-color: #EEF0F5;}
ul li {list-style-image: url(http://www.math.csi.cuny.edu/static/images/julia.png);}  
</style>
\end{html}
\begin{document}

\section{Questions to be handed in for project 4:}

Read about this topic here:
\href{http://mth229.github.io/zeros.html}{Solving for zeros with julia}.

For the impatient, these questions are related to the zeros of a
real-valued function. That is a value $x$ with $f(x)=0$. We will use two
packages: one for plotting and a new one that brings in some
functionality for finding zeros.



\begin{verbatim}
using Gadfly        
using Roots
\end{verbatim}
Graphically, a zero of the function $f(x)$ occurs where the graph
crosses the $x$-axis. Without much work, a zero can be \emph{estimated}
to a few decimal points. For example, we can zoom in on the zero of
$f(x) = x^5 + x - 1$ by graphing over $[0,1]$:



\begin{verbatim}
f(x) = x^5 + x - 1
plot(f, 0, 1)
\end{verbatim}
We can see the answer is near $0.7$. We could re-plot to get closer, but
if more accurate answers are needed, numeric methods are preferred.

The notes talk about a special case - zeros of a polynomial. Due to the
special nature of polynomials, there are special facts known about the
zeros (e.g., the quadratic equation). These can be exploited. The
\texttt{Roots} package provides the \texttt{roots} function to
\emph{numerically} find all the zeros of a polynomial function (real and
complex) and \texttt{fzeros} function to find just the real roots. (The
heavy lifting is done by the \texttt{Polynomials} package.)



\begin{verbatim}
f(x) = x^5 + x - 1
roots(f)
\end{verbatim}


\begin{verbatim}
fzeros(f)
\end{verbatim}
For the general case, the notes mention the bisection method for
zero-finding. This is based on the intermediate value theorem. For this,
one needs a continuous function and a bracketing interval $[a,b]$ (that
is $f(a)$ and $f(b)$ have different signs). Then a zero must exist and
the algorithm finds it by successive division of the interval. The notes
define a \texttt{bisection} method, but more conveniently \texttt{Roots}
implements this in \texttt{fzero}, when called as
\texttt{fzero(f, {[}a,b{]})}.



\begin{verbatim}
f(x) = x^2 - 2
fzero(f, [1,2])         # find sqrt(2)
\end{verbatim}
The \texttt{fzeros} function for general functions needs to have a hint
as to where to search. This call will attempt to find all zeros within
$[-5,5]$ of $f(x)$:



\begin{verbatim}
fzeros(f, [-5,5])
\end{verbatim}
\subsection{Questions to answer}

\subsubsection{Polynomial functions}

\begin{itemize}
\itemsep1pt\parskip0pt\parsep0pt
\item
  Find a zero of the function $f(x) = 216 - 0.65x$.
\end{itemize}

\begin{answer}
    type: numeric
    reminder: zero of f(x) = 216 - 0.65x
    answer: [332.30669230769234, 332.3086923076923]

\end{answer}

\begin{itemize}
\itemsep1pt\parskip0pt\parsep0pt
\item
  The parabola $f(x) = -16x^2 + 200x$ has one zero at $x=0$. Graphically
  find the other one. What is the value
\end{itemize}

\begin{answer}
    type: numeric
    reminder: \(Other root of -16x^2 + 200x\)
    answer: [12.499, 12.501]

\end{answer}

\begin{itemize}
\itemsep1pt\parskip0pt\parsep0pt
\item
  Use the
  \href{http://en.wikipedia.org/wiki/Quadratic_equation}{quadratic
  equation} to find the roots of $f(x) = x^2 + x - 1$. Show your work.
\end{itemize}

\begin{answer}
type: longtext
reminder: quadratic equation to solve \(x^2 + x - 1\)

rows: 3
cols: 60
\end{answer}

\begin{itemize}
\itemsep1pt\parskip0pt\parsep0pt
\item
  Use the \texttt{roots} function to find the zeros of
  $p(x)=x^3 -4x^2 -7x + 10$. What are they?
\end{itemize}

\begin{answer}
type: radio
reminder: zeros of \(p(x)=x^3 -4x^2 -7x + 10\)
values: 1 | 2 | 3
labels: -2.0, 1.0, 5.0 | -0.788376+1.08241im,  -0.788376-1.08241im, 5.57675+0.0im | -2.33333, 0.0
answer: 1
\end{answer}

\begin{itemize}
\itemsep1pt\parskip0pt\parsep0pt
\item
  Use the \texttt{fzeros} function to find the \emph{real} zeros of
  $p(x) = x^5 -5x^4 -2x^3 + 13x^2 -17x + 10$. (The \texttt{fzero}
  function returns all 5 zeros guaranteed by the Fundamental Theorem of
  Algebra, not all of them are real.)
\end{itemize}

\begin{answer}
type: radio
reminder: zeros of \( x^5-5x^4 -2x^3 + 13x^2 -17x + 1\)
values: 1 | 2 | 3 | 4
labels: -2.000 + 0.0im, 0.500 + 0.866im, 0.500 - 0.866im, 1.0 + 0.0im, 5.0 + 0.0im | -2.0, 5.0, 1.0 | 0.0, 2.0, 6.0 | 1.0, 5.0
answer: 2
\end{answer}

\begin{itemize}
\itemsep1pt\parskip0pt\parsep0pt
\item
  Descarte's
  \href{http://en.wikipedia.org/wiki/Descartes_rule_of_signs}{rule of
  signs} allows one to estimate the number of \emph{positive} real roots
  of a real-valued polynomial simply by counting plus and minus signs.
  Write your polynomial with highest powers first and then count the
  number of changes of sign of the coefficients. The number of positive
  real roots is this number or this number minus $2k$ for some $k$.
\end{itemize}

Apply this rule to the polynomial $x^5 - 4x^4 + 5x^3 - 16x^2 -   3$.
What are the possible numbers of positive real roots? What is the exact
number?

The possible number of real roots

\begin{answer}
    type: numeric
    reminder: possible positive roots
    answer: [3, 3]

\end{answer}

The actual number of positive real roots is:

\begin{answer}
    type: numeric
    reminder: actual number of positive roots
    answer: [1, 1]

\end{answer}

\begin{itemize}
\itemsep1pt\parskip0pt\parsep0pt
\item
  The \emph{rational root test} is taught as an aid to factoring. It
  isn't useful numerically, as the answers sought are floating point
  values. One can apply \texttt{rationalize} to any answer to have
  \texttt{julia} approximate a floating point number with a rational
  number. For example,
\end{itemize}



\begin{verbatim}
f(x) = 12x^2 - 7x - 10
map(rationalize, fzeros(f))
\end{verbatim}
Do these roots satisfy the rational root test or are the rational
approximations just approximations?

\begin{answer}
type: radio
reminder: rational root test
values: 1 | 2
labels: Yes they do satisfy the rational root test. For example 2 divides 10 and 3 divides 12 | No they do not satisfy the rational root test. For example 2 divides 12 but 3 \bold{does not} divide 12
answer: 1
\end{answer}

\subsubsection{Other types of functions}

\begin{itemize}
\itemsep1pt\parskip0pt\parsep0pt
\item
  Graph the function $f(x)= x^2 - 2^x$. Try to graphically estimate all
  the zeros. Answers to one decimal point.
\end{itemize}

\begin{answer}
type: radio
reminder: roots of \( x^2 - 2^x \)
values: 1 | 2 | 3 | 4
labels: 0.0, 2.0 | -1.0, 1.0 | 2.0, 4.0 | -1.414, 1.414
answer: 3
\end{answer}

\begin{itemize}
\itemsep1pt\parskip0pt\parsep0pt
\item
  Plot the conversion of Celsius to Fahrenheit ($f(x) = 9/5x + 32$) and
  the line $y=x$. The intersection point is the temperature where both
  systems give the same value. What is the value?
\end{itemize}

\begin{answer}
    type: numeric
    reminder: intersection of F and C
    answer: [-40.1, -39.9]

\end{answer}

\begin{itemize}
\itemsep1pt\parskip0pt\parsep0pt
\item
  Graphically find the point(s) of intersection of the graphs of
  $f(x) = 2.5-   2e^{-x}$ and $g(x) = 1 + x^2$.
\end{itemize}

\begin{answer}
type: longtext
reminder: intersections of \(f(x) = 2.5-2e^{-x}\) and \(g(x) = 1 + x^2\)
answer_text: around 0.4 and 0.7 
rows: 3
cols: 60
\end{answer}

\begin{itemize}
\itemsep1pt\parskip0pt\parsep0pt
\item
  The notes have a \texttt{bisection} method, here is an abbreviated
  version.
\end{itemize}



\begin{verbatim}
function bisection(f, bracket)
  a,b = bracket  
  @assert f(a) * f(b) < 0   # an error if [a,b] is not a bracket
  mid = (a+b)/2

  while a < mid < b
    if f(mid) == 0.0 break end
    f(a) * f(mid) < 0 ? (b = mid) : (a = mid)  
    mid = (a+b) / 2
  end
  mid
end
\end{verbatim}
The function above \textbf{starts} with two values, $a$ and $b$ with the
property that $f(a)$ and $f(b)$ have different signs, hence if $f(x)$ is
continuous, it must cross zero between $a$ and $b$. The algorithm simply
bisects this interval by finding \texttt{mid}. It then selects either
\texttt{{[}a,mid{]}} or \texttt{{[}mid,b{]}} to be the new interval
where a zero is guaranteed. It \textbf{stops} the interval is too small
to subdivide -{}- an impossibility mathematically, but is the case with
floating point numbers.

The \texttt{bisection} function is used to find a zero, when $[a,b]$
bracket a zero for $f$. It is called like
\texttt{bisection(f, {[}a,b{]})}, for a suitable \texttt{f}, \texttt{a},
and \texttt{b}.

Use this function to find a zero of $f(x) = \sin(x)$ on $[3,4]$. Show
both the zero, \texttt{x}, and \texttt{f(x)}.

\begin{answer}
type: longtext
reminder: Commands to find zero of sin(x) in [3,4]

rows: 3
cols: 60
\end{answer}

\begin{itemize}
\itemsep1pt\parskip0pt\parsep0pt
\item
  Let $f(x) = \exp(x) - x^5$. In the long run the exponential dominates
  the polynomial and this function grows unbounded. By graphing over the
  interval $[0,15]$ you can see that the largest zero is less than 15.
  Find a bracket and then use \texttt{bisection} to identify the value
  of the largest zero.
\end{itemize}

\begin{answer}
    type: numeric
    reminder: largest root of \( e^x - x^5 \) in [0,15]
    answer: [12.713106788867632, 12.713306788867632]

\end{answer}

\begin{itemize}
\itemsep1pt\parskip0pt\parsep0pt
\item
  Find the intersection point of $4 - e^{x/10} = e^{x/15}$ by first
  graphing to see approximately where the answer is. From the graph,
  identify a bracket and then use \texttt{fzero} to numerically estimate
  the intersection point.
\end{itemize}

\begin{answer}
    type: numeric
    reminder: largest intersection point of  4 - exp(x/10) = exp(x/15)
    answer: [8.204886667065423, 8.206886667065422]

\end{answer}

\begin{itemize}
\itemsep1pt\parskip0pt\parsep0pt
\item
  The \texttt{Roots} package has a built-in function \texttt{fzero} that
  does different things, with one of them being a (faster) replacement
  for the \texttt{bisection} function. That is, if \texttt{f} is a
  continuous function and \texttt{{[}a,b{]}} a bracketing interval, then
  \texttt{fzero(f, {[}a,b{]})} will find a zero of \texttt{f}.
\end{itemize}

Show that it works by finding a zero of the function
\texttt{f(x) = (1 + (1 - n)\^{}2)*x - (1 - n*x)\^{}2} when $n=10$. Use
$[0, 0.5]$ as a bracketing interval. What is the value?

\begin{answer}
    type: numeric
    reminder: zero of  \( (1 + (1-n)^2)*x - (1-n*x)^2 \)
    answer: [0.0098000099980005, 0.010000009998000499]

\end{answer}

\begin{itemize}
\itemsep1pt\parskip0pt\parsep0pt
\item
  The \texttt{airy} function is a special function of historical
  importance. Find its largest negative zero by first plotting, then
  finding a bracketing interval and finally using \texttt{fzero} to get
  a numeric value.
\end{itemize}

\begin{answer}
    type: numeric
    reminder: largest negative root of airy function
    answer: [-2.338207410459764, -2.3380074104597637]

\end{answer}

\begin{itemize}
\itemsep1pt\parskip0pt\parsep0pt
\item
  Suppose a crisis manager models the number of cases of water left
  after $x$ days by $f(x) = 550,000 \cdot (1 - 0.25)^x$. When does the
  supply of water dip below $100,000$? Find a bracket and then use a
  numeric method to get a precise answer.
\end{itemize}

\begin{answer}
type: longtext
reminder: When does the supply of water dip below 100,000?
answer_text: \(f(x) = 550000*(1-0.25)^x\);\verb+fzero(x->f(x)-100000,[0,10])+; 5.92580 
rows: 3
cols: 60
\end{answer}

\begin{itemize}
\itemsep1pt\parskip0pt\parsep0pt
\item
  Use \texttt{fzeros} to find the roots of $e^x = x^5$ over $[-20,20]$.
\end{itemize}

\begin{answer}
type: radio
reminder: roots of \( e^x - x^5 in [-20,20] \)
values: 1 | 2 | 3 | 4
labels: .129856 | .129856, 12.7132 | 1.85718, 4.53640 | -0.8155, 1.42961, 8.6131
answer: 2
\end{answer}

\subsubsection{Issues with numerics}

\begin{itemize}
\itemsep1pt\parskip0pt\parsep0pt
\item
  Polynomials with either high multiplicity or nearby roots can cause
  issues when seeking numeric approximations. Find the roots of the
  polynomial $f(x) = (x-1)^20$ using \texttt{roots}. What do you see?
  What do you expect? (Compare with \texttt{fzeros}.)
\end{itemize}

\begin{answer}
type: longtext
reminder: what does roots find for \( x-1)^20 \)? What is expected?
answer_text: roots found sit on unit circle. Answer should be just 1.0 
rows: 3
cols: 60
\end{answer}

\begin{itemize}
\itemsep1pt\parskip0pt\parsep0pt
\item
  Consider the polynomial parameterized by \texttt{delta}:
\end{itemize}



\begin{verbatim}
delta = 0.01
f(x) = (x-1-delta)*(x-1)*(x-1+delta)
fzeros(f)
\end{verbatim}
This finds three roots. What happens if \texttt{delta = 0.001}? (The
\texttt{roots} function does better.

\begin{answer}
type: longtext
reminder: What happens when roots are nearby? (delta=0.001)
answer_text: The algorithm only finds two of three 
rows: 3
cols: 60
\end{answer}

\begin{itemize}
\itemsep1pt\parskip0pt\parsep0pt
\item
  The \texttt{fzeros} function for general functions is a very nieve
  function: it simply splits the interval into many pieces and looks for
  bracketing intervals. It will have problems with nearby roots and
  roots which don't have a sign change. Apply it to this function over
  the interval $[-10,10]$. Are any of the three zeros found?
\end{itemize}



\begin{verbatim}
delta = 0.01
f(x) = (x-1-delta)*(x-1)*(x-1+delta) * exp(x)
\end{verbatim}
\begin{answer}
type: longtext
reminder: Are any of the three zeros found?
answer_text: Just one, not 3 as there should be. 
rows: 3
cols: 60
\end{answer}

\begin{itemize}
\itemsep1pt\parskip0pt\parsep0pt
\item
  Will \texttt{fzeros} find all zeros of the function
  $f(x) = \cos(x) + \cos(2x)$ over the interval $[0,2\pi]$?
\end{itemize}

\begin{answer}
type: longtext
reminder: will fzeros find all the zeros?
answer_text: No, it misses the obvious zero at \( pi \) 
rows: 3
cols: 60
\end{answer}

\end{document}

