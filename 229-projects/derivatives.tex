\documentclass[12pt]{article}
\usepackage[fleqn]{amsmath}     %puts eqns to left, not centered
\usepackage{graphicx}
\usepackage{hyperref}
\begin{html}
<style>
pre {font-size: 1.2em; background-color: #EEF0F5;}
ul li {list-style-image: url(http://www.math.csi.cuny.edu/static/images/julia.png);}  
</style>
\end{html}
\begin{document}

\section{Questions to be handed in for project 6:}

To get started, we load \texttt{Gadfly} so that we can make plots and
the \texttt{Roots} package for later usage.



\begin{verbatim}
using Gadfly            
using Roots
\end{verbatim}
\begin{center}\rule{3in}{0.4pt}\end{center}

\subsubsection{Quick background}

Read about this material here:
\href{http://mth229.github.io/derivatives.html}{Approximate derivatives
in julia}.

For the impatient, the slope of the tangent line to the graph of $f(x)$
at the point $(c,f(c))$ is given by the following limit:

\[
\lim_{h \rightarrow 0} \frac{f(c + h) - f(c)}{h}.
\]

The notation for this -{}- when the limit exists -{}- is $f'(c)$, the
intuition is that this is the limit of the slope of a sequence of secant
lines connecting the points $(c, f(c))$ and $(c+h, f(c+h))$. In general
the derivative of a function $f(x)$ is the function $f'(x)$, which
returns the slope of the tangent line for each $x$ where it is defined.

Approximating the tangent line can be done several ways. The
\emph{forward difference quotient} takes a small value of $h$ and uses
the values $(f(x+h) - f(x))/h$ as an approximation. In \texttt{julia} we
can write a function that does this:



\begin{verbatim}
forward(f, x; h=1e-6) = (f(x+h) - f(x))/h
\end{verbatim}
We can define an \emph{operator} -{}- something which takes a function
and returns a function modifying the above slightly:



\begin{verbatim}
Df(f; h=1e-6) = x -> (f(x+h) - f(x))/h
\end{verbatim}
A hack allows us to give this particular operation the prime notation of
\href{http://en.wikipedia.org/wiki/Notation_for_differentiation}{Lagrange}:



\begin{verbatim}
Base.ctranspose(f::Function) = Df(f) 
\end{verbatim}
Then this is used along the lines of (using the default \texttt{h}):



\begin{verbatim}
f(x) = sin(x) + cos(2x)
f'(pi)  
\end{verbatim}
In the \texttt{Roots} package, an operator \texttt{D} (using Euler's
notation) is given which uses a numeric approach to compute the
derivative. This is more accurate, but conceptually a bit more difficult
to understand and does not work for all functions. It is used like an
operator, e.g., \texttt{D(f)} is a function derived from the function
\texttt{f}:



\begin{verbatim}
using Roots
fp(x) = D(sin)(x)       # define a function fp or use D(sin) directly
fp(pi)                  # finds cos(pi). Also D(sin)(pi)
\end{verbatim}
\subsubsection{Questions}

\begin{itemize}
\itemsep1pt\parskip0pt\parsep0pt
\item
  Calculate the slope of the secant line of $f(x) = 3x^2 + 5$ between
  $(2,f(2))$ and $(5, f(5))$.
\end{itemize}

\begin{answer}
    type: numeric
    reminder: slope of secant line
    answer: [13.999, 14.001]

\end{answer}

\begin{itemize}
\itemsep1pt\parskip0pt\parsep0pt
\item
  Verify that the derivative of $f(x) = \sin(x)$ at $\pi/3$ is $1/2$ by
  finding the following limit using a table:
\end{itemize}

\[
\lim_{h \rightarrow 0} \frac{f(\pi/3 + h) - f(\pi/3)}{h}
\]

(Use \texttt{{[}hs ys{]}} to look at your generated data, as was done in
the limits project.)

\begin{answer}
type: longtext
reminder: Verify derivative using a table
answer_text: \verb+[hs map(x->sin(x), hs)]+ 
rows: 3
cols: 60
\end{answer}

\begin{itemize}
\itemsep1pt\parskip0pt\parsep0pt
\item
  Let $f(x) = 1/x$ and $c=2$. Find the approximate derivative (forward)
  when \texttt{h=1e-6}.
\end{itemize}

\begin{answer}
    type: numeric
    reminder: approxderivative \verb+(1/x)'(2)+
    answer: [-0.2500998750238319, -0.24989987502383193]

\end{answer}

\begin{itemize}
\itemsep1pt\parskip0pt\parsep0pt
\item
  Let $f(x) = x^x$ and $c=3$. Find the approximate derivative (forward)
  when \texttt{h=1e-8}.
\end{itemize}

\begin{answer}
    type: numeric
    reminder: approxderivative \verb+(x^x)'(3)+
    answer: [56.662432088580516, 56.66263208858052]

\end{answer}

\begin{itemize}
\itemsep1pt\parskip0pt\parsep0pt
\item
  Let $f(x) = (x + 2)/(1 + x^3)$. Plot both $f$ and its approximate
  derivative on the interval $[0,5]$. Identify the zero of the
  derivative. What is its value? What is the value of $f(x)$ at this
  point?
\end{itemize}

What commands produce the plot?

\begin{answer}
type: longtext
reminder: what commands produce a plot of f and its derivative over [0,5]?

rows: 3
cols: 60
\end{answer}

What is the zero of the derivative on this interval?

\begin{answer}
    type: numeric
    reminder: zero of derivative
    answer: [0.3743671526381416, 0.3943671526381416]

\end{answer}

What is the value of $f$ at this point:

\begin{answer}
    type: numeric
    reminder: value of f(x0)
    answer: [2.2462447684254276, 2.266244768425427]

\end{answer}

\begin{itemize}
\itemsep1pt\parskip0pt\parsep0pt
\item
  Let $f(x) = (x^3 + 5)(x^3 + x + 1)$. The derivative of this function
  has one real zero. Find it. (You can use \texttt{fzero} with the
  derivative function after plotting to identify a bracketing interval.)
\end{itemize}

\begin{answer}
    type: numeric
    reminder: zero of f'(x)
    answer: [-1.3642244478879977, -1.3640244478879977]

\end{answer}

\begin{itemize}
\itemsep1pt\parskip0pt\parsep0pt
\item
  Let $f(x) = \sin(x)$. Following the example on p124 of the Rogawski
  book we look at a table of values of the forward difference equation
  at $x=\pi/6$ for various values of $h$. The true derivative is
  $\cos(\pi/6) = \sqrt{3}/2$.
\end{itemize}

Make the following table. What size \texttt{h} has the closest
approximation?



\begin{verbatim}
f(x) = sin(x)
c = pi/6
hs = [(1/10)^i for i in 1:12]
ys = [forward(f, c, h=h) for h in hs] - sqrt(3)/2
[hs ys]
\end{verbatim}
\begin{answer}
type: radio
reminder: smallest error
values: 1 | 2 | 3 | 4 | 5 | 6 | 7 | 8 | 9 | 10 | 11 | 12
labels: 1e-1 | 1e-2 | 1e-3 | 1e-4 | 1e-5 | 1e-6 | 1e-7 | 1e-8 | 1e-9 | 1e-10 | 1e-11 | 1e-12
answer: 8
\end{answer}

\begin{itemize}
\itemsep1pt\parskip0pt\parsep0pt
\item
  The \texttt{D} operator is easy to use. Here is how we can plot both
  the sine function and its derivative
\end{itemize}



\begin{verbatim}
using Gadfly, Roots         # to load plot and D
f(x) = sin(x)
plot([f, D(f)], 0, 2pi)
\end{verbatim}
Make a plot of $f(x) = \log(x+1) - x + x^2/2$ and its derivative over
the interval $[-3/4, 4]$. The commands are:

\begin{answer}
type: longtext
reminder: Commands to plot f and its derivative
answer_text: \verb+plot([f,D(f)], -3/4, 4)+ 
rows: 3
cols: 60
\end{answer}

Is the derivative always increasing?

\begin{answer}
type: radio
reminder: is derivative increasing?
values: 1 | 2
labels: True | False
answer: 2
\end{answer}

\begin{itemize}
\itemsep1pt\parskip0pt\parsep0pt
\item
  The function $f(x) = x^x$ has a derivative for $x > 0$. Use
  \texttt{fzero} to find a zero of its derivative.
\end{itemize}

\begin{answer}
    type: numeric
    reminder: zero of derivative of \( x^x \)
    answer: [0.36777944117144235, 0.3679794411714423]

\end{answer}

\begin{itemize}
\itemsep1pt\parskip0pt\parsep0pt
\item
  Higher-order derivates can be approximated as well. One can use, say,
  \texttt{D(f,2)} to approximate the second derivative. (The latter uses
  the forward difference approximation twice.) Look carefully at the
  output of these commands. Do they show that the two give similar
  results?
\end{itemize}



\begin{verbatim}
f(x) = sin(x)
g(x) = D(f,2)(x) - f''(x)   # difference of different approximate derivatives
map(g, linspace(0, pi, 10))
\end{verbatim}
Now replace the line \texttt{g(x) = D(f,2)(x) - f''(x)} with a
comparison of 4th derivatives \texttt{g(x) = D(f,4)(x) - f''''(x)}. Do
you get the same level of approximation? In other words, would you want
to use \texttt{f''''(x)} to find a fourth derivative? (Assume
\texttt{D(f,4)} is the correct answer.)

\begin{answer}
type: longtext
reminder: Do D(f,4) and f'''' mostly agree?

rows: 3
cols: 60
\end{answer}

\subsection{Some applications}

\begin{itemize}
\itemsep1pt\parskip0pt\parsep0pt
\item
  Suppose the height of a ball falls according to the formula
  $h(t) =   300 - 16t^2$. Find the rate of change of height at the
  instant the ball hits the ground.
\end{itemize}

\begin{answer}
    type: numeric
    reminder: value of derivative when h is 0
    answer: [-11.772242117486154, -11.772042117486155]

\end{answer}

\begin{itemize}
\itemsep1pt\parskip0pt\parsep0pt
\item
  A formula for blood alcohol level in the body based on time is based
  on the number of drinks and the time
  \href{http://en.wikipedia.org/wiki/Blood_alcohol_content}{wikipedia}.
\end{itemize}

Suppose a model for the number of drinks consumed per hour is



\begin{verbatim}
n(t) = t <= 3 ? 2 * sqrt(3) * sqrt(t) : 6.0
\end{verbatim}
Then the BAL for a 175 pound male is given by



\begin{verbatim}
bal(t) = (0.806 * 1.2 * n(t)) / (0.58 * 175 / 2.2) - 0.017*t
\end{verbatim}
From the plot below, describe when the peak blood alcohol level occurs
and is the person ever in danger of being above 0.10?



\begin{verbatim}
plot([bal, bal'], .5,6)
\end{verbatim}
\begin{answer}
type: longtext
reminder: Describe when peak BAL occurs, is it ever above 0.10?
answer_text: No, it isn't 
rows: 3
cols: 60
\end{answer}

\subsubsection{Tangent lines}

The tangent line to the graph of $f(x)$ at $x=c$ is given by
$y = f(c) + f'(c)(x-c)$. It is fairly easy to plot both the function and
its tangent line. This function is a helper, though it can be done
directly:



\begin{verbatim}
f(x) = x^2          # replace me
tangent(f, c) = x -> f(c) + D(f)(c)*(x-c) # returns a function
plot([f, tangent(f, 1)], 0, 2)
\end{verbatim}
\begin{itemize}
\itemsep1pt\parskip0pt\parsep0pt
\item
  For the function $f(x) = 1/(x^2 + 1)$ (The witch of Agnesi), graph $f$
  over the interval $[-3,3]$ and the tangent line to $f$ at $x=1$.
\end{itemize}

\begin{answer}
type: longtext
reminder: Commands to plot witch of Agnesi over [-3,3] with tangent line at x=1

rows: 3
cols: 60
\end{answer}

\begin{itemize}
\itemsep1pt\parskip0pt\parsep0pt
\item
  Let $f(x) = x^3 -2x - 5$. Find the intersection of the tangent line at
  $x=3$ with the $x$-axis.
\end{itemize}

\begin{answer}
    type: numeric
    reminder: intersection of tangent line with x axis
    answer: [2.3598999999999997, 2.3601]
answer_text: x-f(x)/f'(x) 
\end{answer}

\begin{itemize}
\itemsep1pt\parskip0pt\parsep0pt
\item
  Let $f(x)$ be given by the expression below.
\end{itemize}



\begin{verbatim}
f(x; a=1) = a * log((a + sqrt(a^2 - x^2))/x ) - sqrt(a^2 - x^2)
\end{verbatim}
The value of \texttt{a} is a parameter, for which $a=1$ is fine.

For $x=0.25$ and $x=0.75$ the tangent lines can be drawn with



\begin{verbatim}
plot([f, tangent(f, 0.25), tangent(f, 0.75)], 0.01, 0.8)
\end{verbatim}
Verify that the length of the tangent line between $(c, f(c))$ and the
$y$ axis is the same for $c=.25$ and $c=0.75$. (You can do this by
making an appropriate right triangle whose ratio $opp/adj$ is related to
the derivative, the length is the hypotenuse of this triangle.)

\begin{answer}
type: longtext
reminder: Verify lengths of two lines are same
answer_text: Write a function to compute length squared: \verb#c^2 + (f(c)-(f(c)+D(f)(c)*(0-c)))^2# 
rows: 3
cols: 60
\end{answer}

\end{document}

