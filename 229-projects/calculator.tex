\documentclass[12pt]{article}
\usepackage[fleqn]{amsmath}     %puts eqns to left, not centered
\usepackage{graphicx}
\usepackage{hyperref}
\begin{html}
<style>
pre {font-size: 1.2em; background-color: #EEF0F5;}
ul li {list-style-image: url(http://www.math.csi.cuny.edu/static/images/julia.png);}  
</style>
\end{html}
\begin{document}
\section{Questions to be handed in on Julia as a calculator}\subsubsection{Quick background}\newline
Read about this material here: \href{http://mth229.github.io/calculator.html}{Julia as a calculator}.\newline
For the impatient, these questions cover the use of \texttt{julia} to replace what a calculator can do:\subsubsection{the common operations on numbers: addition, subtraction,  multiplication, division, and powers.}\newline
For the most part there is no surprise, once you learn the notations:   \texttt{+}, \texttt{-}, \texttt{*}, \texttt{/}, and \texttt{^}. (Though you may find that copying and   pasting minus signs will often cause an error, as only something   that looks like a minus sign is pasted in.)\newline
Using \texttt{IJulia}, one types the following into a cell and then presses the \textit{run} button (or \textit{shift-enter}):\begin{verbatim}
2 + 2
\end{verbatim}
\begin{verbatim}
4\end{verbatim}
\newline
The answer follows below the cell.\newline
Here is how one does a slightly more complicated computation:\begin{verbatim}
(2 + 3)^4/(5 + 6)
\end{verbatim}
\begin{verbatim}
56.81818181818182\end{verbatim}
\newline
As with your calculator, it is very important to use parentheses as appropriate to circumvent the usual order of operations.\subsubsection{The use of the basic families of function: trigonometric, exponential, logarithmic.}\newline
On a calculator, there are buttons used to compute various functions. In \texttt{julia}, there are \textit{many} pre-defined functions that serve a similar role (and you will see how to define your own). Functions in \texttt{julia} have names and are called using parentheses to enclose their argument(s), as with:\begin{verbatim}
sin(pi/4), cos(pi/3)
\end{verbatim}
\begin{verbatim}
(0.7071067811865475,0.5000000000000001)\end{verbatim}
\newline
(With \texttt{IJulia}, when a cell is executed only the last command computed is displayed, the above shows that using a comma to separate commands on the same line can be used to get two or more commands to be displayed.)\newline
Most basic functions in \texttt{julia} have easy to guess names, though you will need to learn some differences, such as \texttt{log} is for $\ln$ and \texttt{asin} for $\sin^{-1}$.\subsubsection{the use of memory registers to remember intermediate values.}\newline
Rather than have numbered memory registers, it is \textit{easy} to assign a name to a value. For example,\begin{verbatim}
x = 42
\end{verbatim}
\begin{verbatim}
42\end{verbatim}
\newline
Names can be reassigned (though at times names for functions can not be reassigned to different types of values). For assigning more than one value at once, commas can be used as with the output:\begin{verbatim}
a,b,c = 1,2,3
\end{verbatim}
\begin{verbatim}
(1,2,3)\end{verbatim}
\subsubsection{Julia, like math, has different number types}\newline
Unlike a calculator, but just like math, \texttt{julia} has different types of numbers: integers, rational numbers, real numbers, and complex numbers. For the most part the distinction isn't much to worry about, but there are times where one must, such as overflow with integers. (One can only take the factorial of 20 with 64-bit integers, whereas on most calculators a factorial of 69 can be taken, but not 70.) Julia automatically assigns a type when it parses a value. a \texttt{1} will be an integer, a \texttt{1.0} an floating point number. Rational numbers are made by using two division symbols, \texttt{1//2}.\newline
For many operations the type will be conserved, such as adding to integers. For some operations, the type will be converted, such as dividing two integer values. Mathematically, we know we can divide some integers and still get an integer, but \texttt{julia} usually opts for the same output for its functions (and division is also a function) based on the type of the input, not the values of the input.\newline
Okay, maybe that is too much. Let's get started.\subsubsection{Expressions}\begin{itemize}\item Compute the following value:\end{itemize}
$$
(5/9)(-10 - 32)
$$

\\begin{answer}
    type: numeric
    reminder: \( (5/9) * (-10 - 32) \)
    answer: [-23.334333333333337, -23.332333333333334]
    answer_text: [-23.334, -23.332] 
\\end{answer}
\begin{itemize}\item  Compute the following value:\end{itemize}
$$
9/5(100) + 32
$$

\\begin{answer}
    type: numeric
    reminder: \( 9/5(100) + 32 \)
    answer: [211.999, 212.001]
    answer_text: [211.999, 212.001] 
\\end{answer}
\begin{itemize}\item Compute the following value:\end{itemize}
$$
-4.9\cdot 10^2 + 19.6\cdot 10 + 58.8 
$$

\\begin{answer}
    type: numeric
    reminder: \( -4.9\cdot 10^2 + 19.6\cdot 10 + 58.8 \)
    answer: [-235.20100000000005, -235.19900000000004]
    answer_text: [-235.201, -235.199] 
\\end{answer}
\begin{itemize}\item Compute the following value:\end{itemize}
$$
\frac{1 + 2\cdot 3}{4 + 5^6}
$$

\\begin{answer}
    type: numeric
    reminder: \( \frac{1 + 2cdot 3}{4 + 5^6} \)
    answer: [-0.0005521146586473862, 0.0014478853413526138]
    answer_text: [-0.001, 0.001] 
\\end{answer}
\subsection{Math functions}\begin{itemize}\item Compute the following value:\end{itemize}
$$
\sqrt{0.25\cdot(1-0.25)/100}
$$

\\begin{answer}
    type: numeric
    reminder: \( \sqrt{0.25\cdot(1-0.25)/100} \)
    answer: [0.04230127018922193, 0.044301270189221933]
    answer_text: [0.042, 0.044] 
\\end{answer}
\begin{itemize}\item Compute the following value (here math notation and computer notation are not the same):\end{itemize}
$$
\cos^2(\pi/3)
$$

\\begin{answer}
    type: numeric
    reminder: \( \cos^2(\pi/3) \)
    answer: [0.2490000000000001, 0.2510000000000001]
    answer_text: [0.249, 0.251] 
\\end{answer}
\begin{itemize}\item Compute the following value:\end{itemize}
$$
\sin^2(\pi/3)  \cdot \cos((\pi/6)^2)
$$

\\begin{answer}
    type: numeric
    reminder: \( \sin^2(\pi/3)  \cdot \cos((\pi/6)^2) \)
    answer: [0.7209905957444216, 0.7229905957444216]
    answer_text: [0.721, 0.723] 
\\end{answer}
\begin{itemize}\item Compute the following value:\end{itemize}
$$
e^{(1/2)\cdot(3 - 2.3)^2}
$$

\\begin{answer}
    type: numeric
    reminder: \( e^{(1/2)\cdot(3 - 2.3)^2} \)
    answer: [1.276621313204887, 1.2786213132048867]
    answer_text: [1.277, 1.279] 
\\end{answer}
\begin{itemize}\item Compute the following value:\end{itemize}
$$
1 + \frac{1}{2} + \frac{1}{2\cdot 3} + \frac{1}{2\cdot 3\cdot4} + \frac{1}{2\cdot 3\cdot4\cdot5}
$$

\\begin{answer}
    type: numeric
    reminder: \(1 + \frac{1}{2} + \frac{1}{2\cdot 3} + \frac{1}{2\cdot 3\cdot4} + \frac{1}{2\cdot 3\cdot4\cdot5}\)
    answer: [1.715666666666667, 1.7176666666666667]
    answer_text: [1.716, 1.718] 
\\end{answer}
\begin{itemize}\item Compute the following value (\texttt{cosd} takes degree arguments, \texttt{cos} takes radian values):\end{itemize}
$$
\frac{5}{\cos(57^\circ)}  + \frac{8}{\sin(57^\circ)}
$$

\\begin{answer}
    type: numeric
    reminder: \( \frac{5}{\cos(57^\circ)}  + \frac{8}{\sin(57^\circ)} \)
    answer: [18.718298636570893, 18.720298636570895]
    answer_text: [18.718, 18.72] 
\\end{answer}
\begin{itemize}\item In mathematics a function is defined not only by a rule but also by   a \textit{domain} of possible values. Similarly with \texttt{julia}. What kind of   error does \texttt{julia} respond with if you try this command: \texttt{sqrt(-1)}?\end{itemize}
\\begin{answer}
type: shorttext
reminder: What kind of error of `sqrt(-1)`?
answer: DomainError

\\end{answer}
\subsection{Precedence}\begin{itemize}\item There are 5 operations in the following expression. Write a similar   expression using 4 pairs of parentheses that evaluates to the same   value:\end{itemize}
$$
1 - 2 + 3 \cdot 4 ^ 5 / 6
$$

\\begin{answer}
type: shorttext
reminder: \( 1 - 2 + 3 * 4 ^ 5 / 6 \)
answer: (1 - 2) + ((3 * (4 ^ 5)) / 6)
answer_text: \( (1 - 2) + ((3 * (4 ^ 5)) / 6) \) 
\\end{answer}
\begin{itemize}\item Which of these will also produce $1/(3\cdot4)$:\end{itemize}\newline
\texttt{*} \texttt{1 / 3 * 4}\newline
\texttt{*} \texttt{1 / 3 / 4}\newline
\texttt{*} \texttt{1 * 3 / 4}?
\\begin{answer}
type: radio
reminder: \verb+1/(3cdot4)+
values: 2 | 3 | 1
labels: \verb+1*3/4+ | \verb+1/3*4+ | \verb+1/3/4+
answer: 3

\\end{answer}
\subsection{Variable}\begin{itemize}\item Let \texttt{x=4} and \texttt{y=7} compute\end{itemize}
$$
x - \sin(x + y)/\cos(x - y)
$$

\\begin{answer}
    type: numeric
    reminder: x - sin(x + y)/ cos(x - y)
    answer: [2.9889012265400097, 2.9909012265400094]
    answer_text: [2.989, 2.991] 
\\end{answer}
\begin{itemize}\item For the polynomial\end{itemize}
$$
y = ax^2 + bx + c
$$
\newline
Let $a=0.00014$, $b=0.61$, $c=649$, and $x=200$. What is $y$?
\\begin{answer}
    type: numeric
    reminder: ax^2 + bx + c
    answer: [776.5999, 776.6001]
    answer_text: [776.6, 776.6] 
\\end{answer}
\begin{itemize}\item If \end{itemize}
$$
\frac{\sin(\theta_1)}{v_1} = \frac{\sin(\theta_2)}{v_2}
$$
\newline
and $\theta_1 = \pi/5$, $\theta_2 = \pi/6$, and $v_1=2$, find $v_2$.
\\begin{answer}
    type: numeric
    reminder: find \(v_2\)
    answer: [1.7012016167040798, 1.7014016167040797]
    answer_text: [1.701, 1.701] 
\\end{answer}
\subsection{Some applications}\begin{itemize}\item The period of simple pendulum depends on a gravitational constant   $g=9.8$ and the pendulum length, $L$, in meters, according to the formula:   $T=2\pi\sqrt{L/g}$.\end{itemize}\newline
A rope swing is timed to have a period of $6$ seconds. How long is   the length of the rope if the formula applies?
\\begin{answer}
    type: numeric
    reminder: Find L
    answer: [8.936428397254193, 8.936628397254193]
    answer_text: [8.936, 8.937] 
\\end{answer}
\begin{itemize}\item An object dropped from a building of height $h$ (in feet) will fall   according to the laws of projectile motion:\end{itemize}
$$
y(t) = h - 16t^2
$$
\newline
If $h=50$ find $y$ if $t=1.5$.
\\begin{answer}
    type: numeric
    reminder: find y
    answer: [13.9999, 14.0001]
    answer_text: [14.0, 14.0] 
\\end{answer}
\begin{itemize}\item Suppose $v = 2\cdot 10^8$ and $c = 3 \cdot 10^8$ compute\end{itemize}
$$
\frac{1}{\sqrt{1 - v^2/c^2}}
$$
\newline
(Be careful, this expression from a theory of relativity is susceptible to \textit{integer} overflow on some computers!)
\\begin{answer}
    type: numeric
    reminder: \( \frac{1}{\sqrt{1 - v^2/c^2}} \)
    answer: [1.3415407864998738, 1.3417407864998738]
    answer_text: [1.342, 1.342] 
\\end{answer}
\subsection{Trig practice}\begin{itemize}\item A triangle has sides $a=500$, $b=750$ and $c=901$. Is this a right triangle?\end{itemize}
\\begin{answer}
type: radio
reminder: is this a right triangle?
values: 1 | 2
labels: true | false
answer: 2

\\end{answer}
\begin{itemize}\item The law of sines states for a triangle with angle $A$, $B$, and $C$ and opposite sides labeled $a$, $b$, $c$ one has \end{itemize}
$$ 
\sin(A)/a = \sin(B)/b = \sin(C)/c.
$$
\newline
If $A=115^\circ$, $a=123$, and $b=16$, find $B$ (in degrees).
\\begin{answer}
    type: numeric
    reminder: find B
    answer: [6.770457323410265, 6.770657323410265]
    answer_text: [6.77, 6.771] 
\\end{answer}
\begin{itemize}\item The law of cosines generalizes Pythagorean's theorem: $c^2 = a^2 +   b^2 - 2ab \cos(C)$. A triangle has sides $a=5$, $b=9$, and   $c=8$. Find the angle $C$.\end{itemize}
\\begin{answer}
    type: numeric
    reminder: find C
    answer: [1.0851782044993055, 1.0853782044993054]
    answer_text: [1.085, 1.085] 
\\end{answer}
\subsection{Numbers}\newline
Scientific notation represents real numbers as $a \cdot 10^b$, where $b$ is an integer, and $a$ may be a real number in the range $-1$ to $1$. In \texttt{julia} such numbers are represented with an \texttt{e} to replace the 10, as with \texttt{1.2e3} which would be $1.2 \cdot 10^3$ (1,230) or \texttt{3.2e-1}, which would be $3.2 \cdot 10^{-1}$ (0.32).\begin{itemize}\item What is the sum of \texttt{12e3} and \texttt{32e-1}?\end{itemize}
\\begin{answer}
    type: numeric
    reminder: 
    answer: [12003.199, 12003.201000000001]
    answer_text: [12003.199, 12003.201] 
\\end{answer}
\begin{itemize}\item The output of \texttt{sin(pi)} in \texttt{julia} gives \texttt{1.2246467991473532e-16}. Is this number\end{itemize}
\\begin{answer}
type: radio
reminder: 
values: 3 | 1 | 2
labels: close to -1.22 | close to 0 | close to 1.22
answer: 2

\\end{answer}
\begin{itemize}\item  Is \texttt{7e-10}  greater than \texttt{8e-9}?\end{itemize}
\\begin{answer}
type: radio
reminder: Is 7e-10 greater 8e-9?
values: 1 | 2
labels: true | false
answer: 2

\\end{answer}
\newline
Julia has different storage type for integers (which are stored exactly, but have smaller bounds on their size); rational numbers (which are stored exactly in terms of a numerator and a denominator); real numbers (which are \textit{approximated} by floating point numbers); and complex numbers (which may have either have integer or floating point values for the two components.) When \texttt{julia} parses a value, it will determine the type by how it is entered.\begin{itemize}\item For example, the values \texttt{2}, \texttt{2.0}, \texttt{2 + 0im} and \texttt{2//1} are all the same and yet all different. What type is each?\end{itemize}
\\begin{answer}
type: longtext
reminder: The values 2, 2.0, 2 + 0im and 2//1 are all the same and yet all different. How so?
answer_text: different storage types 
rows: 3
cols: 60
\\end{answer}
\begin{itemize}\item Compute \end{itemize}
$$
2^{-1}.
$$
\newline
(This isn't quite as easy as it looks, as the output of the power function (\texttt{^}) depends on the type of the input variable.)\newline
What command did you use:
\\begin{answer}
type: longtext
reminder: Compute `2^(-1)`
answer_text: either `1/2`, `1/2^1`, `2.0^(-1)`, ..., but not `2^(-1)` 
rows: 3
cols: 60
\\end{answer}

\end{document}
