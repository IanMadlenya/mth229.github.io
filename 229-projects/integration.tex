\documentclass[12pt]{article}
\usepackage[fleqn]{amsmath}     %puts eqns to left, not centered
\usepackage{graphicx}
\usepackage{hyperref}
\begin{html}
<style>
pre {font-size: 1.2em; background-color: #EEF0F5;}
ul li {list-style-image: url(http://www.math.csi.cuny.edu/static/images/julia.png);}  
</style>
\end{html}
\begin{document}

\section{Questions to be handed in on integration:}

To get started, we load \texttt{Gadfly} so that we can make plots, and
load the \texttt{Roots} package for \texttt{D}:



\begin{verbatim}
using Gadfly
using Roots         # for D and fzero
using SymPy
\end{verbatim}
\subsubsection{Quick background}

Read about this material here:
\href{http://mth229.github.io/integration.html}{integration}.

For the impatient, in many cases, the task of evaluating a definite
integral is made easy by the fundamental theorem of calculus which says
that for a continuous function $f$ that the following holds for any of
its antiderivatives, $F$:

\[
\int_a^b f(x) dx = F(b) - F(a).
\]

That is the definite integral is found by evaluating a related function
at the endpoints, $a$ and $b$.

The \texttt{SymPy} package can compute many antiderivatives using a
version of the Risch algorithm that works for \emph{elementary
functions}. Its \texttt{integrate} function can be used to find an
indefinite integral:



\begin{verbatim}
f(x) = x^2
integrate(f)
\end{verbatim}
Or a definite integral:



\begin{verbatim}
integrate(f, 0, 1)      # returns a "symbolic" number
\end{verbatim}
However, this only works \emph{if} there is a known antiderivative
$F(x)$ -{}- which is not always the case. If not, what to do?

In this case, we can appeal to the definition of the definite integral.
For continuous, non-negative $f(x)$, the definite integral is the area
under the graph of $f$ over the interval $[a,b]$. This area can be
directly \emph{approximated} using Riemann sums, or some other
approximation scheme.

Imagine one didn't know, $F$. Then the Riemann approximation can help.
The following pattern will compute the sum using right-hand endpoints:



\begin{verbatim}
f(x) = x^2
a, b, n = 0, 1, 5       # 5 partitions of [0,1] requested
delta = (b - a)/n       # size of partition
xs = a + (1:n) * delta  
fxs = [f(x) for x in xs]
sum(fxs * delta)        # a new function `sum` to add up values in a container
\end{verbatim}
That isn't very close to $1/3$. But we only took $n=5$. Bigger $n$s mean
better approximations:



\begin{verbatim}
f(x) = x^2
a, b, n = 0, 1, 50_000      # 50,000 partitions of [0,1] requested
delta = (b - a)/n       
xs = a + (1:n) * delta  
fxs = [f(x) for x in xs]
sum(fxs * delta)
\end{verbatim}
Note that only the first two lines need changing to adjust to a new
problem. This makes it easy to wrap this in a function. We borrow this
more elaborate one from the notes:



\begin{verbatim}
function riemann(f::Function, a::Real, b::Real, n::Int; method="right")
  if method == "right"
     meth(f,l,r) = f(r) * (r-l)
  elseif method == "left"
     meth(f,l,r) = f(l) * (r-l)
  elseif method == "trapezoid"
     meth(f,l,r) = (1/2) * (f(l) + f(r)) * (r-l)
  elseif method == "simpsons"
     meth(f,l,r) = (1/6) * (f(l) + 4*(f((l+r)/2)) + f(r)) * (r-l)
  end

  xs = a + (0:n) * (b-a)/n
  as = [meth(f, l, r) for (l,r) in zip(xs[1:end-1], xs[2:end])]
  sum(as)
end
\end{verbatim}
The Riemann sum is very slow to converge here. There are faster
algorithms both mathematically and computationally. We will discuss two:
the trapezoid rule, which replaces rectangles with trapezoids; and
Simpson's rule which is a quadratic approximation.



\begin{verbatim}
f(x) = x^2
riemann(f, 0, 1, 1000, method="trapezoid"), riemann(f, 0, 1, 1000, method="simpsons")
\end{verbatim}
Base \texttt{julia} provides the \texttt{quadgk} function which uses a
different approach altogether. It is used quite easily:



\begin{verbatim}
f(x) = x^2
ans, err = quadgk(f, 0, 1)
\end{verbatim}
This function returns two values, an answer and an estimated maximum
possible error. The ans is the first number, clearly it is $1/3$, and
the estimated maximum error is the second. In this case it is small
($10^{-17}$) and is basically 0.

\subsubsection{Questions}

\begin{itemize}
\itemsep1pt\parskip0pt\parsep0pt
\item
  Let $g(x) = x^4 + 10x^2 - 60x + 71$. Find the integral
  $\int_0^1   g(x) dx$ exactly using the fundamental theorem of
  calculus.
\end{itemize}

\begin{answer}
type: longtext
reminder: Commands to do FTC
answer_text: \verb# G(x) = x^5/5 + 10x^3/3 -60x^2/2+71x; G(1)-G(0)=44.53 
rows: 3
cols: 60
\end{answer}

\begin{itemize}
\itemsep1pt\parskip0pt\parsep0pt
\item
  For $f(x) = x/\sqrt{g(x)}$ (for $g(x)$ from the last problem) estimate
  the following using 1000 Riemann sums:
\end{itemize}

\[
\int_0^1 f(x) dx
\]

\begin{answer}
    type: numeric
    reminder: riemann sum, n=1000
    answer: [0.15622112243469194, 0.15622114243469193]

\end{answer}

\begin{itemize}
\itemsep1pt\parskip0pt\parsep0pt
\item
  Let $f(x) = \sin(\pi x^2)$. Estimate $\int_0^1 f(x) dx$ using 20
  Riemann sums
\end{itemize}

\begin{answer}
    type: numeric
    reminder: riemann sum, n=20
    answer: [0.503543430844665, 0.5035434508446651]

\end{answer}

\begin{itemize}
\itemsep1pt\parskip0pt\parsep0pt
\item
  For the same $f(x)$, compare your estimate with 20 Riemann sums to
  that with 20,000 Riemann sums. How many digits after the decimal point
  do they agree?
\end{itemize}

\begin{answer}
type: radio
reminder: Where do they differ?
values: 1 | 2 | 3 | 4 | 5 | 6
labels: They differ at the first place after the decimal point | They differ at the second place after the decimal point | They differ at the third place after the decimal point | They differ at the fourth place after the decimal point | They differ at the fifth place after the decimal point | They differ at the sixth place after the decimal point
answer: 3
\end{answer}

\paragraph{Left Riemann}

The left Riemann sum uses left-hand endpoints, not right-hand ones.

\begin{itemize}
\itemsep1pt\parskip0pt\parsep0pt
\item
  For $f(x) = e^{x}$ use the left Riemann sum with $n=10,000$ to
  estimate $\int_0^1 f(x) dx$.
\end{itemize}

\begin{answer}
    type: numeric
    reminder: left-hand intervals
    answer: [1.7181959057993523, 1.7181959257993522]

\end{answer}

\begin{itemize}
\itemsep1pt\parskip0pt\parsep0pt
\item
  The left and right Riemann sums for an increasing function are also
  lower and upper bounds for the answer. Find the difference between the
  left and right Riemann sum for $\int_0^1 e^x dx$ when $n=10,000$. (Use
  your last answer.) What is the approximate value $1/100$, $1/1000$,
  $1/10000$, or $1/100000$?
\end{itemize}

\begin{answer}
type: radio
reminder: Approximate difference for 10,000 steps
values: 1 | 2 | 3 | 4
labels: 1/100 | 1/1000 | 1/10000 | 1/100000
answer: 3
\end{answer}

\paragraph{trapezoid, Simpson's}

\begin{itemize}
\itemsep1pt\parskip0pt\parsep0pt
\item
  The answer to $\int_0^1 e^x dx$ is simply $e^1 - e^0$ = $e-1$. Compare
  the error (in absolute value) of the trapezoid method when $n=10,000$.
\end{itemize}

\begin{answer}
type: radio
reminder: Size of error
values: 1 | 2 | 3 | 4 | 5 | 6 | 7
labels: The error is about 1e-6 | The error is about 1e-7 | The error is about 1e-8 | The error is about 1e-9 | The error is about 1e-10 | The error is about 1e-11 | The error is about 1e-12
answer: 4
\end{answer}

\begin{itemize}
\itemsep1pt\parskip0pt\parsep0pt
\item
  The answer to $\int_0^1 e^x dx$ is simply $e^1 - e^0$ = $e-1$. Compare
  the error of the Simpson's method when $n=10,000$.
\end{itemize}

\begin{answer}
type: radio
reminder: Size of error, simpsons
values: 1 | 2 | 3 | 4 | 5 | 6 | 7
labels: The error is about 1e-10 | The error is about 1e-11 | The error is about 1e-12 | The error is about 1e-13 | The error is about 1e-14 | The error is about 1e-15 | The error is about 1e-16
answer: 4
\end{answer}

(The error for Riemann sums is like $1/n$, the error for trapezoid sums
is $1/n^2$ and that for Simpson's is $1/n^4$.)

\subsection{quadgk}

\begin{itemize}
\itemsep1pt\parskip0pt\parsep0pt
\item
  Use \texttt{quadgk} to find $\int_{-3}^{3} (1 + x^2)^{-1} dx$. What is
  the answer? What is the estimated maximum error?
\end{itemize}

The answer is:

\begin{answer}
    type: numeric
    reminder: area under f
    answer: [2.49809153479651, 2.4980915547965097]

\end{answer}

The error is about

\begin{answer}
type: radio
reminder: Size of error, quadgk
values: 1 | 2 | 3 | 4 | 5 | 6 | 7
labels: The error is about 1e-6 | The error is about 1e-7 | The error is about 1e-8 | The error is about 1e-9 | The error is about 1e-10 | The error is about 1e-11 | The error is about 1e-12
answer: 3
\end{answer}

\begin{itemize}
\itemsep1pt\parskip0pt\parsep0pt
\item
  Use \texttt{quadgk} to find the integral of $e^{-\vert x\vert }$ over
  $[-1,1]$.
\end{itemize}

\begin{answer}
    type: numeric
    reminder: area under f
    answer: [1.2642411076571154, 1.2642411276571153]

\end{answer}

\subsubsection{Improper integrals}

An \emph{improper integral} is one which involves infinity one of few
ways:

\begin{enumerate}
\def\labelenumi{\arabic{enumi})}
\item
  one or both limits is infinite
\item
  the function $f$ has a vertical asymptote in the interval $[a,b]$
  (e.g., $f(x) = 1/x$ on $[0,1]$.
\end{enumerate}

For these cases, the fundamental theorem of calculus does not apply, but
the definite integral can be defined in terms of limiting values over
sub-ranges. (For example,
$\lim_{M} \int_0^M e^{-x} dx = \int_0^\infty e^{-x} dx$.)

\href{http://www.hpl.hp.com/hpjournal/pdfs/IssuePDFs/1980-08.pdf}{Kahan}
(in a very interesting article about integration on a calculator) goes
on to add these as troubling:

\begin{enumerate}
\def\labelenumi{\arabic{enumi})}
\setcounter{enumi}{2}
\item
  the integrand oscillates infinitely rapidly in the interval $[a,b]$
  (e.g., $f(x) = \sin(1/x)$ on $[-1,1]$.
\item
  the integrand or its first derivative changes wildly within a
  relatively narrow subinterval or oscillates frequently.
\end{enumerate}

The \texttt{quadgk} function can handle these cases well, as we see
through some examples

\begin{itemize}
\itemsep1pt\parskip0pt\parsep0pt
\item
  The integral $\int_0^1 1/\sqrt{x}dx$ is an improper integral that is
  defined. What is the value?
\end{itemize}

\begin{answer}
    type: numeric
    reminder: 1/sqrt(x) from 0 to 1
    answer: [1.9999999745983916, 1.9999999945983915]

\end{answer}

Why is this integral "improper?"

\begin{answer}
type: radio
reminder: Why improper
values: 1 | 2 | 3
labels: The domain to integrate over is infinite | There is a vertical asymptote | The integral oscillates wildly
answer: 2
\end{answer}

\begin{itemize}
\itemsep1pt\parskip0pt\parsep0pt
\item
  The integral $\int_0^1 x^{-2} dx$ is an improper integral that is not
  defined. How does \texttt{julia} report the error?
\end{itemize}

\begin{answer}
type: shorttext
reminder: How is the error reported?
answer: DomainError
answer_text: It throws a domain error 
\end{answer}

Why is this integral "improper?"

\begin{answer}
type: radio
reminder: Why improper
values: 1 | 2 | 3
labels: The domain to integrate over is infinite | There is a vertical asymptote | The integral oscillates wildly
answer: 2
\end{answer}

\begin{itemize}
\itemsep1pt\parskip0pt\parsep0pt
\item
  The integral $\int_0^\infty e^{-x^2} dx$ is important in probability
  theory and many other areas. Compute its value with \texttt{quadgk}.
  (\texttt{Inf} is infinity.)
\end{itemize}

\begin{answer}
    type: numeric
    reminder: integral \( exp(-x^2) \)
    answer: [0.8862259254527578, 0.8862279254527579]

\end{answer}

Why is this integral "improper?"

\begin{answer}
type: radio
reminder: Why improper
values: 1 | 2 | 3
labels: The domain to integrate over is infinite | There is a vertical asymptote | The integral oscillates wildly
answer: 1
\end{answer}

\begin{itemize}
\itemsep1pt\parskip0pt\parsep0pt
\item
  The function $f(x) = x \cdot \sin(1/x)$ must be redefined to have a
  limit at $0$. The integral $\int_0^1 x \cdot \sin(1/x)dx$ is defined.
  What is the value estimated by \texttt{quadgk}?
\end{itemize}

\begin{answer}
    type: numeric
    reminder: integral of x*sin(1/x)
    answer: [0.37852901682944334, 0.3785310168294433]

\end{answer}

Why is this integral "improper?" (In Kahan's widening of the sense.)

\begin{answer}
type: radio
reminder: Why improper
values: 1 | 2 | 3
labels: The domain to integrate over is infinite | There is a vertical asymptote | The integral oscillates wildly
answer: 3
\end{answer}

\begin{itemize}
\itemsep1pt\parskip0pt\parsep0pt
\item
  Define
\end{itemize}

\[
f(u) = \frac{\sqrt{u}}{u-1} - \frac{1}{\log{u}}
\]

The improper integral $\int_0^1 f(u) du$ is defined. What is the value?
What is the estimated error? Is this consistent with a value of
$0.03649$ give or take \texttt{1e-10}?

\begin{answer}
type: radio
reminder: Consistent with the value
values: 1 | 2
labels: yes | no
answer: 1
\end{answer}

\subsection{Applications}

We discuss an application of the integral to finding the volumes -{}-
not just areas. The first is a formula for a solid of revolution. For
cylindrically symmetric volume, such as a drinking glass, if the radius
as a function of height is given by $r(h)$, the the volume is
$\int_a^b \pi r(h)^2 dh$.

\subsubsection{glass half full}

\begin{itemize}
\itemsep1pt\parskip0pt\parsep0pt
\item
  A glass is formed as a volume of revolution with radius as a function
  of height given the equation $r(h) = 2 + (h/20)^4$. The volume as a
  function of height $b$ is given by $V(b) = \int_0^b \pi   r(h)^2 dh$.
  Find $V(25)$. Show your work.
\end{itemize}

\begin{answer}
type: longtext
reminder: Volume of glass
answer_text: \verb#R(h) = 2 + (h/20)^4;V(b) = quadgk(x->pi*R(x)^2, 0, b)[1];V(25)# 519... 
rows: 3
cols: 60
\end{answer}

\begin{itemize}
\itemsep1pt\parskip0pt\parsep0pt
\item
  Find a value of $b$ so that $V(b) = 455$.
\end{itemize}

\begin{answer}
type: longtext
reminder: b so that V(b) = 455
answer_text: \verb#fzero(b->V(b)-455, 25)# 23.85 
rows: 3
cols: 60
\end{answer}

\begin{itemize}
\itemsep1pt\parskip0pt\parsep0pt
\item
  Now find a value of $b$ for which $V(b) = 455/2$. This height will
  have half the volume as the height just found. Compare the two values.
  Is the ratio of smallest to largest 1/2, more than 1/2 or less?
\end{itemize}

\begin{answer}
type: longtext
reminder: b so that V(b) = 455/2
answer_text: \verb#fzero(b->V(b)-455/2, 25/2)# 16.48 
rows: 3
cols: 60
\end{answer}

\end{document}

