\documentclass[12pt]{article}
\usepackage[fleqn]{amsmath}     %puts eqns to left, not centered
\usepackage{graphicx}
\usepackage{hyperref}
\begin{html}
<style>
pre {font-size: 1.2em; background-color: #EEF0F5;}
ul li {list-style-image: url(http://www.math.csi.cuny.edu/static/images/julia.png);}  
</style>
\end{html}
\begin{document}
\section{Questions to be handed in on Newton's Method:}\newline
Begin by loading our package for plotting and the Roots package\begin{verbatim}
using Plots
gadfly()
using Roots
\end{verbatim}
\rule{\textwidth}{1pt}
\subsubsection{Quick background}\newline
Read about this material here: \href{http://mth229.github.io/newton.html}{Newton's Method}.\newline
For the impatient, symbolic math $-$ as is done behind the scenes at the Wolfram alpha web site $-$ is pretty nice. For so many problems it can easily do what is tedious work. However, for some questions, only numeric solutions are possible. For example, there is no general formula to solve a fifth order polynomial the way there is a quadratic formula for solving quadratic polynomials. Even an innocuous polynomial like $f(x) = x^5 - x - 1$ has no easy algebraic solution:\begin{verbatim}
using SymPy
@vars x
\end{verbatim}
\begin{verbatim}
using SymPy
x = symbols("x")
\end{verbatim}
$$x$$\begin{verbatim}
solve(x^5 - x - 1)
\end{verbatim}
\begin{bmatrix}\operatorname{RootOf} {\left(x^{5} - x - 1, 0\right)}\\\operatorname{RootOf} {\left(x^{5} - x - 1, 1\right)}\\\operatorname{RootOf} {\left(x^{5} - x - 1, 2\right)}\\\operatorname{RootOf} {\left(x^{5} - x - 1, 3\right)}\\\operatorname{RootOf} {\left(x^{5} - x - 1, 4\right)}\end{bmatrix}\newline
We see that \texttt{SymPy} basically punts on this question.\newline
Numeric solutions are available. As this is a polynomial, we could use the \texttt{roots} function for \texttt{Roots}:\begin{verbatim}
f(x) = x^5 - x - 1
roots(f)
\end{verbatim}
\begin{verbatim}
5-element Array{Complex{Float64},1}:
    1.1673+0.0im     
  0.181232+1.08395im 
  0.181232-1.08395im 
 -0.764884+0.352472im
 -0.764884-0.352472im\end{verbatim}
\newline
We see 5 roots, as expected from a fifth degree polynomial, with one real root (the one with \texttt{0.0im}) that is approximately 1.1673. Finding such a value usually requires some iterative root-finding algorithm (though not in the case above which uses linear algebra). For polynomials, the \texttt{fzeros} function uses such an algorithm \textit{for polynomials} to find the real roots:\begin{verbatim}
fzeros(f)                # no a, b range needed for polynomials.
\end{verbatim}
\begin{verbatim}
1-element Array{Real,1}:
 1.1673\end{verbatim}
\newline
Newton's method is a root-finding algorithm.  Like the bisection method discussed earlier, it is an \textit{iterative algorithm}. The algorithm starts with some \textit{guess} for a \textit{zero} to an equation $f(x)=0$. If this guess is called $x_0$, then the algorithm gives a \textit{new (and improved!)} guess $x_1$. It is expected that $x_1$ is a better guess, but may not be the best that can be. The algorithm is then repeated \textit{again} to produce $x_2$. This is done until some guess, $x_n$, is as close enough or the algorithm fails for some reason. The \textit{approximate zero} is taken to be $x_n$.\newline
What is the algorithm for Newton's method? It is simple. If we start with some $x_i$, then $x_{i+1}$ is given by the intersection point of the $x$-axis of the tangent line of $f(x)$ at $x_i$. Mathematically then we can equate our two methods to compute the slope of a tangent line:
$$
f'(x_i) = \frac{f(x_i) - 0}{x_i - x_{i+1}}
$$
\newline
Or, solving for $x_{i+1}$:
$$
x_{i+1} = x_i - f(x_i)/f'(x_i)
$$
\newline
Let's see this algorithm for $f(x) = x^3−2x−5$, a function that Newton himself considered. He was looking for a solution near $2$, so let's start there:\begin{verbatim}
x = 2
f(x) = x^3 - 2x -5
fp(x) = 3x^2 - 2		# done by hand
\end{verbatim}
\begin{verbatim}
fp (generic function with 1 method)\end{verbatim}
\newline
We don't need to track the index ($x_0$, $x_1$, ...), as when we write the following expression, the next value is just assigned to \texttt{x} using the \textit{current} value of \texttt{x} when computed:\begin{verbatim}
x = x - f(x) / fp(x)
x, f(x)				# display both the new guess, x,  and the value f(x)
\end{verbatim}
\begin{verbatim}
(2.1,0.06100000000000083)\end{verbatim}
\newline
The value of $2.1$ is a better guess, but not near the actual answer. We simply repeat to (hopefully) get a better guess:\begin{verbatim}
x = x - f(x) / fp(x)
x, f(x)
\end{verbatim}
\begin{verbatim}
(2.094568121104185,0.00018572317327247845)\end{verbatim}
\newline
Here are a few more repeats:\begin{verbatim}
x = x - f(x) / fp(x)
x, f(x)
\end{verbatim}
\begin{verbatim}
(2.094551481698199,1.7397612239733462e-9)\end{verbatim}
\begin{verbatim}
x = x - f(x) / fp(x)
x, f(x)
\end{verbatim}
\begin{verbatim}
(2.0945514815423265,-8.881784197001252e-16)\end{verbatim}
\newline
The value of \texttt{f(x)} is now \textit{basically} 0, and any further updates to \texttt{x} do not change its value. We see that the algorithm has converged to an answer, \texttt{x}, and the fact that it is a zero is confirmed by the value of \texttt{f(x)}.\newline
Repeating steps in \texttt{IJulia} can be a bit of a chore. There a few  ways to make this easier. For example, encapsulate the algorithm into a function or use a programming construct to repeat the task.\newline
For the former, you might have:\begin{verbatim}
newt(x, f, fp) = x - f(x)/fp(x)
\end{verbatim}
\begin{verbatim}
newt (generic function with 1 method)\end{verbatim}
\newline
and then for a given f, do something like\begin{verbatim}
f(x) =  x^3 - 2x -5; fp(x) = 3x^2 - 2
newt(x) = newt(x, f, fp)
x = 2.0
newt(newt(newt(newt(newt(x)))))
\end{verbatim}
\begin{verbatim}
2.0945514815423265\end{verbatim}
\newline
That is kinda ugly. Is this form any better?\begin{verbatim}
x |> newt |> newt |> newt |> newt |> newt
\end{verbatim}
\begin{verbatim}
2.0945514815423265\end{verbatim}
\newline
Here we use a programming construct, a \textit{macro}, to repeat some \textit{expression} 5 times. (This macro basically replaces the expression internally with 5 repeats of the expression.)\begin{verbatim}
macro take5(body) quote Float64[$(esc(body)) for _ in 1:5] end end # take5 macro
\end{verbatim}
\newline
Macros are prefaced with a \texttt{@} in their name and are typically called without parentheses:\begin{verbatim}
x = 2				# starting value
@take5     x = x - f(x) / fp(x)
\end{verbatim}
\begin{verbatim}
5-element Array{Float64,1}:
 2.1    
 2.09457
 2.09455
 2.09455
 2.09455\end{verbatim}
\newline
and to see that \texttt{x} has been updated we have:\begin{verbatim}
x, f(x)
\end{verbatim}
\begin{verbatim}
(2.0945514815423265,-8.881784197001252e-16)\end{verbatim}
\subsubsection{Questions}\begin{itemize}\item Apply Newton's Method to the function $f(x) = \sin(x)$ with an   initial guess $3$. (This was historically used to compute many   digits of $\pi$ efficiently.) What is the answer after 5 iterations?   What is the value of \texttt{sin} at the answer?\end{itemize}\newline
The value of $f(x)$ after 5 iterations is:
\\begin{answer}
    type: numeric
    reminder: 5 iterations of Newton's method for sin(x), x0=3
    answer: [3.141592643589793, 3.141592663589793]
    answer_text: [3.142, 3.142] 
\\end{answer}
\newline
The value of $f(x)=\sin(x)$ at this approximate zero:
\\begin{answer}
    type: numeric
    reminder: value of f at approximate zero
    answer: [-9.999999999987754e-5, 0.00010000000000012247]
    answer_text: [-0.0, 0.0] 
\\end{answer}
\begin{itemize}\item Use Newton's method to find a zero for the function   $f(x)=x^5-x-1$. Start at $x=1.6$. What is the approximate root after   5 iterations? What is the value of $f(x)$ for your answer? If you do   one or two more iterations, will your guess be better?\end{itemize}\newline
The value after 5 iterations
\\begin{answer}
    type: numeric
    reminder: 5 iterations of Newton's method for \( x^5 -x - 1 \), x0=1.6
    answer: [1.167203979733408, 1.167403979733408]
    answer_text: [1.167, 1.167] 
\\end{answer}
\newline
The value of $f(x)$:
\\begin{answer}
    type: numeric
    reminder: value of f(x)
    answer: [2.1930383567558918e-9, 2.2193038356755892e-8]
    answer_text: [0.0, 0.0] 
\\end{answer}
\newline
The value after two more iterations:
\\begin{answer}
    type: numeric
    reminder: 7 iterations of Newton's method for \( x^5 -x - 1 \), x0=1.6
    answer: [1.1673039682614188, 1.1673039882614187]
    answer_text: [1.167, 1.167] 
\\end{answer}
\begin{itemize}\item Use Newton's method to find a zero of the function $f(x) = \cos(x) -   x$. Make a graph to identify an initial guess.\end{itemize}\newline
Show your commands below
\\begin{answer}
type: longtext
reminder: Commands to find the zero of \( \cos(x) -x \)

rows: 3
cols: 60
\\end{answer}
\newline
The value of the approximate zero:
\\begin{answer}
    type: numeric
    reminder: zero of cos(x) - x
    answer: [0.7380851332151607, 0.7400851332151607]
    answer_text: [0.738, 0.74] 
\\end{answer}
\subsubsection{Using D for the derivative}\newline
If the function \texttt{f(x)} allows it, the \texttt{D} operator from the \texttt{Roots} package can simplify the Newton's method algorithm, as the derivative need not be computed by hand. In this case, the algorithm in \texttt{julia} becomes \texttt{x = x - f(x)/D(f)(x)}.\begin{itemize}\item Use Newton's method to find an intersection point of $f(x) =   e^{-x^2}$ and $g(x)=x$. (Look at $h(x) = f(x) - g(x) = 0$.) Start   with a guess of $0$.\end{itemize}
\\begin{answer}
    type: numeric
    reminder: x-value of intersection point
    answer: [0.6529186403192047, 0.6529186405192047]
    answer_text: [0.653, 0.653] 
\\end{answer}
\begin{itemize}\item Use Newton's method to find \textit{both} positive intersection points of   $f(x) = e^x$ and $g(x) = 2x^2$. Make a graph to identify good   initial guesses.\end{itemize}\newline
The smallest value is:
\\begin{answer}
    type: numeric
    reminder: smaller zero
    answer: [1.487952065498177, 1.4879720654981772]
    answer_text: [1.488, 1.488] 
\\end{answer}
\newline
The largest value is:
\\begin{answer}
    type: numeric
    reminder: larger zero
    answer: [2.6178566130668126, 2.6178766130668127]
    answer_text: [2.618, 2.618] 
\\end{answer}
\subsubsection{Using newton and fzero from the Roots package}\newline
The \texttt{newton} function in the \texttt{Roots} package will compute newton's method. For example:\begin{verbatim}
f(x) = sin(x)
fp(x) = cos(x)
x = 3
newton(f, fp, x)
\end{verbatim}
\begin{verbatim}
3.141592653589793\end{verbatim}
\newline
The extra argument \texttt{verbose=true} will show the iterations:\begin{verbatim}
newton(f, fp, 3, verbose=true)
\end{verbatim}
\begin{verbatim}
3.141592653589793\end{verbatim}
\newline
However, the \texttt{fzero} function $-$ that we have seen before $-$ will use a derivative-free algorithm, similar to Newton's method to find a zero. So, the above zero can \textit{also} be found with:\begin{verbatim}
fzero(sin, 3)
\end{verbatim}
\begin{verbatim}
3.141592653589793\end{verbatim}
\newline
(That is right, \texttt{fzero} can be used two different ways – at least. Above it is called with an initial guess. Previously, we called it with a bracketing interval, as in \texttt{fzero(sin, [3,4])}. If you specify a bracketing interval, \texttt{fzero} will use an algorithm guaranteed to converge. If you just specify an initial guess, the convergence is generally faster, but may not happen.)\rule{\textwidth}{1pt}
\begin{itemize}\item find a zero of $f(x) = x\cdot (2+\ln(x))$ starting at $1$. What is   your answer? How small is the function for this value?\end{itemize}\newline
What is the value of the zero?
\\begin{answer}
    type: numeric
    reminder: zero of x * ( 2 + log(x)) starting at 1
    answer: [0.1353342832366127, 0.1353362832366127]
    answer_text: [0.135, 0.135] 
\\end{answer}
\newline
The value of the function at the zero?
\\begin{answer}
    type: numeric
    reminder: f(xstar)
    answer: [-1.0e-8, 1.0e-8]
    answer_text: [-0.0, 0.0] 
\\end{answer}
\begin{itemize}\item Use \texttt{fzero} to find all zeros of the function $f(x) = 2\sin(x) -   \cos(2x)$ in $[0, 2\pi]$. (Graph first to see approximate answers.)\end{itemize}
\\begin{answer}
type: longtext
reminder: Zeros of 2sin(x) - cos(2x) on [0,2pi]
answer_text: \verb+[fzero(f, x) for x in [0.5, 3]]+ gives .375, 2.767 
rows: 3
cols: 60
\\end{answer}
\begin{itemize}\item Use \texttt{fzero} to find when the derivative of $f(x) = 5/\cos(x) +   7/\sin(x)$ is $0$ in the interval $(0, \pi/2)$.\end{itemize}
\\begin{answer}
    type: numeric
    reminder: zero of derivative of \( 5/cos(x) + 7/sin(x) \)
    answer: [0.841349666225046, 0.8413696662250459]
    answer_text: [0.841, 0.841] 
\\end{answer}
\subsubsection{When Newton's method fails}\newline
Newton's method can fail due to various cases:\newline
\texttt{*} the initial guess is not close to the zero\newline
\texttt{*} the derivative, $|f'(x)|$ is too small\newline
\texttt{*} the second derivative $|f''(x)|$ is too big, or possibly undefined\newline
\texttt{*} Earlier  the roots of $f(x) = x^5 - x - 1$ were considered. Try   Newton's method with an initial guess of $x_0=0$ to find a real   root. Why does this fail? (You can look graphically. Otherwise, you   could look at the output of \texttt{newton} with this extra argument:   \texttt{newton(f, fp, x0, verbose=true)}.
\\begin{answer}
type: longtext
reminder: Why does Newton's method fail?
answer_text: Too far from x0 (on other side of min) 
rows: 3
cols: 60
\\end{answer}
\begin{itemize}\item Let \texttt{f(x) = abs(x)^(1/3)}. Starting at \texttt{x=1}, Newton's method will   fail to converge. What happens? Are any of the above 3 reason's to   blame?\end{itemize}
\\begin{answer}
type: longtext
reminder: Why does Newton's method fail?
answer_text: f'' doesn't exist, basically too big 
rows: 3
cols: 60
\\end{answer}
\subsubsection{Quadratic convergence}\newline
When Newton's method converges to a \textit{simple zero} it is said to have \textit{quadratic convergence}. A simple zero is one with multiplicity 1 and quadratic convergence says basically that the error at the $i+1$st step is like the error for $i$th step squared. In particular, if the error is like $10^{-3}$ on one step, it will be like $10^{-6}$, then $10^{-12}$ then $10^{-24}$ on subsequent steps. (Which is typically beyond the limit of a floating point approximation.) This is why one can \textit{usually} take just 5, or so, steps to get to an answer.\newline
Not so for multiple roots. \begin{itemize}\item For the function \texttt{f(x) = (8x*exp(-x^2) -2x - 3)^8}, starting with   \texttt{x=-2.0}, Newton's method will converge, but it will take many steps   to get to an answer that has $f(x)$ around $10^{-16}$. How many?   Roughly how many iterations do you need? (A single call of    \texttt{@take5 x = x-f(x)/D(f)(x)} gives an answer with \texttt{f(x) = 0.00028} only.)\end{itemize}
\\begin{answer}
type: radio
reminder: How many steps to get convergence of newton's method?
values: 1 | 2 | 3 | 4 | 5 | 6 | 7 | 8
labels: about 5 steps | about 10 steps | about 15 steps | about 20 steps | about 25 steps | about 30 steps | about 35 steps | about 40 steps
answer: 6

\\end{answer}
\begin{itemize}\item Repeat the above with \texttt{f(x) = 8x*exp(-x^2) -2x - 3} $-$ there is no   extra power of $8$ here $-$ and again, starting with   \texttt{x=-2.0}. Roughly how many iterations are needed now?\end{itemize}
\\begin{answer}
type: radio
reminder: How many steps to get convergence of newton's method?
values: 1 | 2 | 3 | 4 | 5 | 6 | 7 | 8
labels: about 5 steps | about 10 steps | about 15 steps | about 20 steps | about 25 steps | about 30 steps | about 35 steps | about 40 steps
answer: 1

\\end{answer}

\end{document}
