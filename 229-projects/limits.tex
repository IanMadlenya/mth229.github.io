\documentclass[12pt]{article}
\usepackage[fleqn]{amsmath}     %puts eqns to left, not centered
\usepackage{graphicx}
\usepackage{hyperref}
\begin{html}
<style>
pre {font-size: 1.2em; background-color: #EEF0F5;}
ul li {list-style-image: url(http://www.math.csi.cuny.edu/static/images/julia.png);}  
</style>
\end{html}
\begin{document}

\section{Questions to be handed in for project 5 on limits}

To get started, we load \texttt{Gadfly} so that we can make plots.



\begin{verbatim}
using Gadfly        
\end{verbatim}
\begin{center}\rule{3in}{0.4pt}\end{center}

\subsubsection{Quick background}

Read about this material here:
\href{http://mth229.github.io/limits.html}{Investigating limits with
Julia}.

For the impatient, the expression

\[
\lim_{x \rightarrow c} f(x) = L
\]

says that the limit as $x$ goes to $c$ of $f$ is $L$. If $f(x)$ is
\emph{continuous} at $x=c$, the $L=f(c)$. This is almost always the case
for a randomly chosen $c$ - but almost never the case for a textbook
choice of $c$. There invariably -{}- though not always -{}- we will have
\texttt{f(c) = NaN} indicating the function is indeterminate at
\texttt{c}. For such cases we need to do more work to identify if any
such $L$ exists and if does, what its value is.

We can investigate limits three ways: analytically, with a table of
numbers, or graphically. Here we focus on two ways: graphically or
numerically.

Investigating a limit numerically requires us to operationalize the idea
of $x$ getting close to $c$ and $f(x)$ getting close to $L$. The first
is easy: just create numbers getting close to 0:



\begin{verbatim}
xs = [(1/10)^i for i in 1:10]
\end{verbatim}
Then we can investigate limits by looking the corresponding
\texttt{f(x)} values. For example, the limit of $\sin(x)/x$ near $0$ is
investigated with:



\begin{verbatim}
f(x) = sin(x)/x
ys = [f(x) for x in xs]     # y values. Alternatively ys = map(f, xs)
[xs ys]             # arrange in a table
\end{verbatim}
From this we see a \emph{right} limit at 0 appears to be $1$. (Look at
what happens to the column on the right.)

Limits when $c\neq 0$ are similar, but require points getting close to
$c$. For example,

\[
\lim_{x \rightarrow \pi/2} \frac{1 - \sin(x)}{(\pi/2 - x)^2}
\]

has a limit of $1/2$. We can investigate with:



\begin{verbatim}
c = pi/2
xs = [c + (1/10)^i for i in 1:10]
f(x) = (1 - sin(x))/(pi/2 - x)^2
ys = map(f, xs)
[xs ys]
\end{verbatim}
Wait, is the limit $1/2$ or $0$? Here we see a limitation of tables -{}-
when numbers get too small, floating point issues creep in. In this
case, when the difference of \texttt{x} from \texttt{pi/2} gets to be
\texttt{1e-8} the difference between \texttt{1} and \texttt{sin(x)} is
too small to have a floating point number other than \texttt{1}
represent \texttt{sin(x)}, hence the numerator becomes $0$. So watch out
when seeing what the values of $f(x)$ get close to. Here it is clear
that the limit is heading towards $0.5$ until we get too close.

For convenience, this function can make the above easier to do:



\begin{verbatim}
function lim(f::Function, c::Real; n::Int=6, dir="+")
     hs = [(1/10)^i for i in 1:n] # close to 0
     if dir == "+"
       xs = c + hs 
     else
       xs = c - hs
     end
     ys = map(f, xs)
     [xs ys]
end
\end{verbatim}
A graphical approach doesn't give as many significant digits, but won't
have this floating point error. Here is a graph to investigate the same
problem. We simply graph near $c$ and look:



\begin{verbatim}
plot(f, c - pi/6, c + pi/6)
\end{verbatim}
From the graph, we see clearly that as $x$ is close to $\pi/2$, $f(x)$
is close to $1/2$. (Gadfly either doesn't include a point for \texttt{c}
or ignores that value when present by treating it like \texttt{NaN}.)

\subsection{Questions: Graphical approach}

\begin{itemize}
\itemsep1pt\parskip0pt\parsep0pt
\item
  Plot the function to estimate the limit. What is the value?
\end{itemize}

\[
\lim_{\theta \rightarrow 0} \frac{\sin(5\theta)}{\sin(2\theta)}.
\]

\begin{answer}
    type: numeric
    reminder: limit of sin(5t)/sin(2t)
    answer: [2.4, 2.6]

\end{answer}

\begin{itemize}
\itemsep1pt\parskip0pt\parsep0pt
\item
  Plot the function to estimate the limit. What is the value?
\end{itemize}

\[
\lim_{\theta \rightarrow 0} \frac{2^x - \cos(x)}{x}.
\]

\begin{answer}
    type: numeric
    reminder: limit of \( (2^x  - cos(x))/x \)
    answer: [0.5931471805599453, 0.7931471805599453]

\end{answer}

\begin{itemize}
\itemsep1pt\parskip0pt\parsep0pt
\item
  Plot the function to estimate the limit. What is the value?
\end{itemize}

\[
\lim_{\theta \rightarrow 0} \frac{\sin^2(4\theta)}{\cos(\theta) - 1}.
\]

\begin{answer}
    type: numeric
    reminder: limit of \( sin(4x)^2 / (cos(x) - 1) \)
    answer: [-32.1, -31.9]

\end{answer}

\subsection{Questions: Tables}

\begin{itemize}
\itemsep1pt\parskip0pt\parsep0pt
\item
  These expressions are all \emph{indeterminate} at $c=0$ (that is, the
  limit -{}- should it exist -{}- is not found by continuity). Which
  return \texttt{NaN} if you try to evaluate them at $0$?
\end{itemize}

\[
\frac{1-\cos(x)}{x}, \quad, x^{1/\log(x)}, \quad, \frac{\tan(x)}{x},\quad x\log(x).
\]

\begin{answer}
type: longtext
reminder: Which return NaN?
answer_text: All but \( x^{1/\log(x)} \) 
rows: 3
cols: 60
\end{answer}

\begin{itemize}
\itemsep1pt\parskip0pt\parsep0pt
\item
  Let $c=0$. Which of these functions is indeterminate at $x=c$:
\end{itemize}

\begin{enumerate}
\def\labelenumi{\arabic{enumi})}
\itemsep1pt\parskip0pt\parsep0pt
\item
  $f(x) = \sin(x)/x$
\item
  $f(x) = \sqrt{x}/(x-2)$
\item
  $f(x) = x^x$
\item
  $f(x) = (1 + 1/x)^x$.
\end{enumerate}

\begin{answer}
type: checkbox
reminder: which are indeterminate?
values: 1 | 2 | 3 | 4
labels: \( f(x) = \sin(x)/x \) | \( f(x) = \sqrt{x}/(x-2) \) | \( f(x) = x^x \) | \( f(x) = (1 + 1/x)^x \)
answer: 1 | 2 | 4
\end{answer}

\begin{itemize}
\itemsep1pt\parskip0pt\parsep0pt
\item
  Find the limit using a table. Show your commands.
\end{itemize}

\[
\lim_{x \rightarrow 0+} \frac{\cos(x) - 1}{x}.
\]

\begin{answer}
type: longtext
reminder: Commands for a table to investigate limit of (cos(x) -1)/x

rows: 3
cols: 60
\end{answer}

What is the estimated value?

\begin{answer}
    type: numeric
    reminder: estimate limit
    answer: [-0.01, 0.01]

\end{answer}

\begin{itemize}
\itemsep1pt\parskip0pt\parsep0pt
\item
  Find the limit using a table. What is the estimated value?
\end{itemize}

\[
\lim_{x \rightarrow 0+} \frac{\sin(5x)}{x}.
\]

\begin{answer}
    type: numeric
    reminder: estimate limit
    answer: [4.99, 5.01]

\end{answer}

\begin{itemize}
\itemsep1pt\parskip0pt\parsep0pt
\item
  Find the limit using a table. What are your commands? What is the
  estimated value? (You need values getting close to $3$ not $0$.)
\end{itemize}

\[
\lim_{x \rightarrow 3} \frac{x^3 - 2x^2 -9}{x^2 - 2x -3}.
\]

The commands are:

\begin{answer}
type: longtext
reminder: Command to find limit at 3 of \( \frac{x^3 - 2x^2 -9}{x^2 - 2x -3} \)

rows: 3
cols: 60
\end{answer}

The value is

\begin{answer}
    type: numeric
    reminder: limit of \( \frac{x^3 - 2x^2 -9}{x^2 - 2x -3} \)
    answer: [3.74000000000000, 3.76000000000000]

\end{answer}

\begin{itemize}
\itemsep1pt\parskip0pt\parsep0pt
\item
  Find the limit using a table. What is the estimated value?
\end{itemize}

\[
\lim_{x \rightarrow 0+} \frac{x - \sin(\vert x\vert )}{x^3}.
\]

\begin{answer}
    type: numeric
    reminder: limit of \( (x - sin(abs(x)))/x^3 \)
    answer: [0.15666666666666665, 0.17666666666666667]

\end{answer}

\begin{itemize}
\itemsep1pt\parskip0pt\parsep0pt
\item
  Find the \emph{left} limit of \texttt{f(x) = cos(pi/2*(x - floor(x)))}
  as $x$ goes to $2$.
\end{itemize}

\begin{answer}
    type: numeric
    reminder: limit of f(x) = cos(pi/2*(x - floor(x)))
    answer: [-0.001, 0.001]

\end{answer}

\begin{itemize}
\itemsep1pt\parskip0pt\parsep0pt
\item
  Find the limit using a table. What is the estimated value? Recall,
  \texttt{atan} and \texttt{asin} are the names for the appropriate
  inverse functions.
\end{itemize}

\[
\lim_{x \rightarrow 0+} \frac{\tan^{-1}(x) - 1}{\sin^{-1}(x) - 1}.
\]

\begin{answer}
    type: numeric
    reminder: limit of   (atan(x) - 1)/(asin(x) - 1)
    answer: [0.99, 1.01]

\end{answer}

\subsection{Other questions}

\begin{itemize}
\itemsep1pt\parskip0pt\parsep0pt
\item
  Let \texttt{f(x) = sin(sin(x)\^{}2) / x\^{}k}. Consider $k=1$, $2$,
  and $3$. For which of values of $k$ does the limit at $0$ \textbf{not}
  exist? (Repeat the problem for the 3 different values.)
\end{itemize}

\begin{answer}
type: longtext
reminder: For which values does the limit not exist?
answer_text: k=3 
rows: 3
cols: 60
\end{answer}

\begin{itemize}
\itemsep1pt\parskip0pt\parsep0pt
\item
  Let \texttt{l(x) = (a\^{}x - 1)/x} and \emph{define}
  $L(a) = \lim_{x\rightarrow 0} l(x,a)$.
\end{itemize}

Show that $L(3 \cdot 4) = L(3) + L(4)$ by computing all three limits
numerically. (In general, you can show algebraically that
$L(a\cdot b) = L(a) + L(b)$ like a logarithm.

\begin{answer}
type: longtext
reminder: Show that \( L(3 cdot 4) = L(3) + L(4) \) by computing all three limits numerically

rows: 3
cols: 60
\end{answer}

\subsection{Limits at infinity.}

By using x values which grow large, we can get a sense of a limit as $x$
goes to infinity. (This requires a different definition that the
$\epsilon$-$\delta$ limit.) For example, this suggests the limit of
$f(x) = \sin(x)/x$ at infinity is $0$:



\begin{verbatim}
xs = [10.0^i for i in 1:8]
f(x) = sin(x)/x
ys = [f(x) for x in xs]
[xs ys]
\end{verbatim}
\begin{itemize}
\itemsep1pt\parskip0pt\parsep0pt
\item
  Find the limit at infinity of $f(x) = x^5/e^x$ (Exponential functions
  with $a>1$ eventually grow faster than polynomials).
\end{itemize}

\begin{answer}
    type: numeric
    reminder: limit at oo of \( x^5/e^x \)
    answer: [-0.01, 0.01]

\end{answer}

\begin{itemize}
\itemsep1pt\parskip0pt\parsep0pt
\item
  Find the limit at infinity of $f(t) = e^t / (1 + e^t)$.
\end{itemize}

\begin{answer}
    type: numeric
    reminder: limit at oo of \( e^t / (1 + e^t) \)
    answer: [0.99, 1.01]

\end{answer}

\subsection{Symbolic limits}

The add-on package \texttt{SymPy} can compute the limit of a simple
algebraic function of a single variable quite well.



\begin{verbatim}
using SymPy
\end{verbatim}


\begin{verbatim}
f(x) = sin(x)/x
limit(f, 0)
\end{verbatim}
\begin{itemize}
\itemsep1pt\parskip0pt\parsep0pt
\item
  Find this limit using \texttt{SymPy} (use a decimal value for your
  answer, not a fraction):
\end{itemize}

\[
\lim_{x \rightarrow 3} \frac{1/x - 1/3}{x^2 - 9}.
\]

\begin{answer}
    type: numeric
    reminder: limit at 3 of \( (1/x - 1/3) / (x^2 - 9) \)
    answer: [-0.02851851851851852, -0.008518518518518517]

\end{answer}

\begin{itemize}
\itemsep1pt\parskip0pt\parsep0pt
\item
  Find this limit using \texttt{SymPy}:
\end{itemize}

\[
\lim_{x \rightarrow 0} \frac{\sin^{-1}(4x)}{\sin^{-1}(5x)}.
\]

\begin{answer}
    type: numeric
    reminder: limit at 0 of \( asin(4x)/asin(5x) \)
    answer: [-0.02851851851851852, -0.008518518518518517]

\end{answer}

\begin{itemize}
\itemsep1pt\parskip0pt\parsep0pt
\item
  Find this limit using \texttt{SymPy}:
\end{itemize}

\[
\lim_{x \rightarrow 0} \frac{\sin(x^2)}{x\tan(x)}.
\]

\begin{answer}
    type: numeric
    reminder: limit at 0 of \( sin(x^2)/(x*tan(x)) \)
    answer: [0.99, 1.01]

\end{answer}

\end{document}

