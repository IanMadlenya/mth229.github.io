\documentclass[12pt]{article}
\usepackage[fleqn]{amsmath}     %puts eqns to left, not centered
\usepackage{graphicx}
\usepackage{hyperref}
\begin{html}
<style>
pre {font-size: 1.2em; background-color: #EEF0F5;}
ul li {list-style-image: url(http://www.math.csi.cuny.edu/static/images/julia.png);}  
</style>
\end{html}
\begin{document}

\section{Questions to be handed in for project 3:}

Read about this here:
\href{http://mth229.github.io/graphing.html}{Graphing Functions with
Julia}.

For the impatient, \texttt{julia} has several packages that allow for
graphical presentations, but nothing "built-in." We will use the
\texttt{Gadfly} package for producing graphics. Assuming it has been
installed, we need to load it into a session as follows:



\begin{verbatim}
using Gadfly 
\end{verbatim}
This brings in a \texttt{plot} function that makes plotting functions as
easy as specifying a function object and the $x$ limits to plot over:



\begin{verbatim}
plot(sin, 0, 2pi)       # plot(f, a, b)
\end{verbatim}
Often most of the battle is \emph{judiciously} choosing the values of
$a$ and $b$ so that the graph highlights a feature of interest (a zero,
a relative peak or valley, an intersection point, ...).

A plot is nothing more than a connect-the-dot graph of $x$ and $y$
values. It can be useful to know how to do the steps. The above could be
done with:



\begin{verbatim}
a, b = 0, 2pi
xs = linspace(a, b)     # 100 points between a and b
ys = [sin(x) for x in xs]   # of ys = map(sin, xs) (see the notes)
plot(x=xs, y=ys, Geom.line) # Geom.line connects the points. Try without to see.
\end{verbatim}
The \texttt{xs} and \texttt{ys} are written as though they are "plural"
because these variables contain 100 values each in a container (a vector
in this case). Containers (vectors in this case) are often constructed
by combining like values within square brackets separated by commas:
e.g., \texttt{{[}a,b{]}}. For plotting, we can combine functions using
\texttt{{[}{]}} and all will plot:



\begin{verbatim}
plot([sin, cos], 0, 2pi)
\end{verbatim}
\begin{center}\rule{3in}{0.4pt}\end{center}

\subsubsection{Questions}

\begin{itemize}
\itemsep1pt\parskip0pt\parsep0pt
\item
  Make a plot of $f(x) = \exp(x) - x^3$ over the interval $[3,5]$.
  Estimate the value where the graph crosses the $x$ axis.
\end{itemize}

The commands to produce the plot are:

\begin{answer}
type: longtext
reminder: commands to plot \( e^x - x^3 \)

rows: 3
cols: 60
\end{answer}

The approximate zero is:

\begin{answer}
    type: numeric
    reminder: approximate zero
    answer: [4.5, 4.699999999999999]
answer_text: 4.6 
\end{answer}

\begin{itemize}
\itemsep1pt\parskip0pt\parsep0pt
\item
  For the same function $f(x) = \exp(x) - x^3$ make graphs over
  different domains until you can find another zero. What is its
  approximate value?
\end{itemize}

\begin{answer}
    type: numeric
    reminder: second approximate zero
    answer: [1.7, 1.9000000000000001]
answer_text: 1.8 
\end{answer}

\begin{itemize}
\itemsep1pt\parskip0pt\parsep0pt
\item
  Graph the function $f(x) = 2x^3 - 5x^2 + x$. By graphing different
  domains, approximate the location of the three zeros.
\end{itemize}

The smallest root is:

\begin{answer}
    type: numeric
    reminder: smallest
    answer: [-0.001, 0.001]

\end{answer}

The middle root is:

\begin{answer}
    type: numeric
    reminder: middle
    answer: [0.21822359359558463, 0.22022359359558463]

\end{answer}

The largest root is:

\begin{answer}
    type: numeric
    reminder: largest
    answer: [-0.001, 0.001]

\end{answer}

\begin{itemize}
\itemsep1pt\parskip0pt\parsep0pt
\item
  A cell phone plan has 700 minutes of talking for 20 dollars with each
  additional minute over 700 minutes costing 10 cents per minute. Write
  a function representing this rate for any positive time $t$. Then
  graph the function between $0$ and $1000$.
\end{itemize}

\begin{answer}
type: longtext
reminder: Commands to plot cell phone plan
answer_text: f(x) = x < 700 ? 20.0  : 20.0 + 0.10*(x-700); plot(f, 0, 1000) 
rows: 3
cols: 60
\end{answer}

\begin{itemize}
\itemsep1pt\parskip0pt\parsep0pt
\item
  Graph the rational function $f(x) = (x^2 + 1)/ (x - 1)$. Do you see
  any asymptotes? If so, describe them.
\end{itemize}

\begin{answer}
type: longtext
reminder: Describe asymptotes of \(f(x) = (x^2 + 1)/ (x - 1) \)
answer_text: One slant with slope 1, one vertical at x=1 
rows: 3
cols: 60
\end{answer}

\begin{itemize}
\itemsep1pt\parskip0pt\parsep0pt
\item
  Make a graph of the rational function
  $f(x) = (x^2 - 2x + 1)/(x^2 -   4)$. Use a suitable domain so that any
  horizontal asymptotes can be seen. What commands did you use?
\end{itemize}

\begin{answer}
type: longtext
reminder: Commands to plot \( f(x) = (x^2 - 2x + 1)/(x^2 - 4) \) showing aymptotes

rows: 3
cols: 60
\end{answer}

\begin{center}\rule{3in}{0.4pt}\end{center}

The following function can be used to restrict the range of a
mathematical function:



\begin{verbatim}
trim(f::Function; cutoff=10) = x -> abs(f(x)) > cutoff ? NaN : f(x)
\end{verbatim}
\begin{itemize}
\itemsep1pt\parskip0pt\parsep0pt
\item
  Try plotting \texttt{trim(f)} when $f(x) = (x^2 - 2x + 1)/(x^2 - 4)$
  over $[-3, 3]$. What do you see as compared to the previous graph of
  $f(x)$?
\end{itemize}

\begin{answer}
type: longtext
reminder: What do you see with using \verb+trim+?
answer_text: The vertical asymptotes do not distort the view 
rows: 3
cols: 60
\end{answer}

\begin{itemize}
\itemsep1pt\parskip0pt\parsep0pt
\item
  Make a plot of $f(x) = \cos(x)$ and $g(x) = 1 - x^2/2$ over
  $[-\pi/2, \pi/2]$. How many times to the graphs intersect? Can you
  even tell? If not, why not?
\end{itemize}

\begin{answer}
type: longtext
reminder: How many times to the graphs intersect? Can you even tell? If not, why not?
answer_text: Can't distinguish the graph near 0. 
rows: 3
cols: 60
\end{answer}

\begin{itemize}
\itemsep1pt\parskip0pt\parsep0pt
\item
  Make a plot of \texttt{f(x) = max(0, 1-abs(x))} and
  \texttt{g(x) = 1 +   2*f(x-3)}. Describe the relationship of
  \texttt{g} and \texttt{f} in terms of the values $1$, $2$ and $3$.
\end{itemize}

\begin{answer}
type: longtext
reminder: Describe the relationship of g and f in terms of the values 1,2 and 3
answer_text: graph of g is shift up 1, scaled by 2, shifted right by 3 units 
rows: 3
cols: 60
\end{answer}

\begin{itemize}
\itemsep1pt\parskip0pt\parsep0pt
\item
  Make a plot of \texttt{f(x) = sin(x)} and
  \texttt{g(x) = cos(x) \textgreater{} 0 ? 0.0 : NaN} over $[0, 2\pi]$.
  What is the relationship? (Notice, the graph of $g(x)$ shows only when
  $\cos(x)$ is positive.)
\end{itemize}

\begin{answer}
type: longtext
reminder: What is the relationship between graph of sin and graph of cos when cosine is positive?
answer_text: increasing function, of course 
rows: 3
cols: 60
\end{answer}

\subsection{creating sequences}

(There are two ways to generate sequences of numbers: the range operator
\texttt{a:h:b} and \texttt{linspace}.)

\begin{itemize}
\itemsep1pt\parskip0pt\parsep0pt
\item
  write a simple command to produce the values: $1, 3, 5, \dots, 99$
\end{itemize}



\begin{verbatim}
shortq("1:2:99", "write a simple command to produce the values 1, 3, 5, ..., 99")
\end{verbatim}
\begin{itemize}
\itemsep1pt\parskip0pt\parsep0pt
\item
  write a simple command to produce 100 values between $0$ and $2\pi$
\end{itemize}



\begin{verbatim}
shortq("linspace(0, 2pi)", "write a simple command to produce  100 values between 0 and \\$2\pi\\$")
\end{verbatim}
\subsection{mapping a function}

\begin{itemize}
\itemsep1pt\parskip0pt\parsep0pt
\item
  If \texttt{a = {[}1,2,3,4,5{]}} find \texttt{a\^{}3} for each value.
  (There are many different ways: using \texttt{map}, using a
  comprehension, using a "dot", ...)
\end{itemize}

\begin{answer}
type: longtext
reminder: find \verb+a^3+ for a=[1,2,3,4,5]
answer_text: Use dot, map or comprehension 
rows: 3
cols: 60
\end{answer}

\begin{itemize}
\itemsep1pt\parskip0pt\parsep0pt
\item
  Explain in words what this particular construct is doing:
\end{itemize}



\begin{verbatim}
[x^2 for x in 1:5]
\end{verbatim}
\begin{answer}
type: longtext
reminder: What is \verb+[x^2 for x in 1:5]+?

rows: 3
cols: 60
\end{answer}

\end{document}

